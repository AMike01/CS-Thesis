\cleardoublepage
\phantomsection
\pdfbookmark{Sommario}{Sommario}
\begingroup
\let\clearpage\relax
\let\cleardoublepage\relax
\let\cleardoublepage\relax

\chapter*{Sommario}

Il presente documento descrive il lavoro svolto durante il periodo di stage, della durata di circa trecentoventi ore, dal laureando \myName presso l'azienda Zero12 s.r.l, 
nel periodo compreso tra il 25 settembre ed il 22 novembre 2024. \\
L'obiettivo principale del progetto era quello di andare a realizzare una piattaforma \textit{Web}, che tramite l'utilizzo di sistemi di intelligenza artificiale Generativa, 
fosse in grado di produrre in automatico dei documenti tecnici contenenti la descrizione delle attività di sviluppo di servizi e infrastrutture \textit{cloud}. \\
Il progetto è stato sviluppato seguendo le metodologie \textit{Agile} e \textit{Scrum}, tutte le componenti implementate sono state opportunamente documentate e 
il loro corretto funzionamento è stato testato.\\ 

\subsection*{Struttura del documento}
La seguente relazione è strutturata nei seguenti capitoli:
\begin{description}
    \item[{\hyperref[cap:contesto-aziendale]{\textbf{Capitolo 1, Contesto aziendale:}}}]  Vado a descrivere i prodotti e servizi offerti dall'azienda, la propensione aziendale all'innovazione.
        Descrivo inoltre la metodologia di lavoro utilizzata, gli strumenti e le tecnologie adottate.
    \item[{\hyperref[cap:introduzione-al-progetto]{\textbf{Capitolo 2, Introduzione al progetto di stage:}}}] Vado a descrivere l'idea del progetto di stage, gli obiettivi e i vincoli imposti dall'azienda, oltre alle motivazioni che mi hanno portato a scegliere questo progetto.
    \item[{\hyperref[cap:svolgimento-dello-stage]{\textbf{Capitolo 3, Svolgimento dello stage:}}}] Descrivo le attività svolte durante il periodo di stage e i risultati ottenuti.
    \item[{\hyperref[cap:retrospettiva-finale]{\textbf{Capitolo 4, Retrospettiva finale:}}}] Vado a descrivere gli obiettivi raggiunti e le difficoltà incontrate.
    Valuto inoltre le competenze acquisite ed effettuo un confronto tra università e mondo lavorativo.
\end{description}

Alla fine del documento si trovano le seguenti sezioni:
\begin{itemize}
    \item \textbf{Acronimi e abbreviazioni:} Contiene i collegamenti alle definizioni dei relativi termini contenute nel \textbf{Glossario};
    \item \textbf{Glossario:} Contiene le definizioni dei termini tecnici;
    \item \textbf{Bibliografia:} Contiene i riferimenti bibliografici utilizzati durante la stesura del documento.
\end{itemize}
\pagebreak

\subsection*{Convenzioni tipografiche}
All'interno di questo documento sono andato ad addottare le seguenti convenzioni tipografiche:
\begin{itemize}
    \item Gli acronimi, le abbreviazioni e i termini ambigui o di uso non comune menzionati vengono definiti nel \textbf{Glossario}, situato alla fine del presente documento;
    \item Riporto ogni termine in lingua diversa dall'italiano in \textit{corsivo};
    \item Riporto ogni termine rilevante in \textbf{grassetto}.
\end{itemize}


%\vfill

%\selectlanguage{english}
%\pdfbookmark{Abstract}{Abstract}
%\chapter*{Abstract}

%\selectlanguage{italian}

\endgroup

\vfill
