% Acronyms
\newacronym[description={\glslink{apig}{Application Program Interface}}]
    {api}{API}{\textit{Application Program Interface}}

\newacronym[description={\glslink{umlg}{Unified Modeling Language}}]
    {uml}{UML}{\textit{Unified Modeling Language}}

\newacronym[description={\glslink{awsg}{Amazon Web Services}}]
    {aws}{AWS}{\textit{Amazon Web Services}}

\newacronym[description={\glslink{itsg}{Issue Tracking System}}]
    {its}{ITS}{\textit{Issue Tracking System}}

\newacronym[description={\glslink{ideg}{Integrated Development Environment}}]
    {ide}{IDE}{\textit{Integrated Development Environment}}

\newacronym[description={\glslink{prg}{Pull Request}}]
    {pr}{PR}{\textit{Pull Request}}

\newacronym[description={\glslink{spag}{Single Page Application}}]
    {spa}{SPA}{\textit{Single Page Application}}

\newacronym[description={\glslink{odmg}{Obejct Data Modeling}}]
    {odm}{ODM}{\textit{Obejct Data Modeling}}

\newacronym[description={\glslink{llmg}{Large Language Model}}]
    {llm}{LLM}{\textit{Large Language Model}}

\newacronym[description={\glslink{jwtg}{JSON Web Token}}]
    {jwt}{JWT}{\textit{JSON Web Token}}

\newacronym[description={\glslink{aig}{Artificial Intelligence}}]
    {ai}{AI}{\textit{Artificial Intelligence}}

\newacronym
    {npm}{NPM}{\textit{Node Package Manager}}

\newacronym[description={\glslink{httpg}{Hypertext Transfer Protocol}}]
    {http}{HTTP}{\textit{Hypertext Transfer Protocol}}

\newacronym[description={\glslink{ragg}{Retrieval Augmented Generation}}]
    {rag}{RAG}{\textit{Retrieval Augmented Generation}}

% Glossary entries
\newglossaryentry{apig} {
    name=\glslink{api}{API},
    text=\textit{Application Program Interface},
    sort=api,
    description={Il termine \emph{Application Programming Interface API} indica ogni insieme di procedure disponibili al programmatore, di solito raggruppate a formare un set di strumenti specifici per l'espletamento di un determinato compito all'interno di un certo programma. La finalità è ottenere un'astrazione, di solito tra l'hardware e il programmatore o tra software a basso e quello ad alto livello semplificando così il lavoro di programmazione}
}

\newglossaryentry{umlg} {
    name=\glslink{uml}{UML},
    text=UML,
    sort=uml,
    description={Il termine \emph{UML, Unified Modeling Language} è un linguaggio di modellazione e specifica basato sul paradigma object-oriented. L'\emph{UML} svolge un'importantissima funzione di ``lingua franca'' nella comunità della progettazione e programmazione a oggetti. Gran parte della letteratura di settore usa tale linguaggio per descrivere soluzioni analitiche e progettuali in modo sintetico e comprensibile a un vasto pubblico}
}

\newglossaryentry{awsg}{
    name=\glslink{aws}{AWS},
    text=\textit{Amazon Web Services},
    sort=aws,
    description={\emph{Amazon Web Services} è una piattaforma di servizi cloud offerta da Amazon. Questi servizi sono divisi in diverse categorie, tra cui calcolo, storage, database, analisi, machine learning, sicurezza, sviluppo di applicazioni, mobile, IoT, IA, AR e VR, media e sviluppo di giochi. AWS offre oltre 200 servizi completi da data center in tutto il mondo}
}

\newglossaryentry{user-stories}{
    name=User Stories,
    text=\textit{User Stories},
    sort=User Stories,
    description={Le \textit{User stories} rappresentano una pratica \textit{Agile}, che viene utilizzata per comprendere le esigenze dell’utente, mediante una descrizione informale, semplice e concisa della funzionalità}
}

\newglossaryentry{product-backlog}{
    name=Product Backlog,
    text=\textit{Product Backlog},
    sort=Product Backlog,
    description={Il \textit{Product Backlog} è una lista di tutte le funzionalità, i requisiti, le modifiche e le correzioni che devono essere fatte ad un prodotto. Questa lista è dinamica e viene aggiornata costantemente, in modo da riflettere le esigenze del cliente}
}

\newglossaryentry{itsg}{
    name=\glslink{its}{ITS},
    text=\textit{Issue Tracking System},
    sort=its,
    description={Un \textit{Issue Tracking System} è un software che permette di gestire e monitorare le attività, i problemi e le richieste di assistenza dei clienti. Questo sistema permette di tenere traccia di tutte le attività, assegnarle ai membri del team e monitorare il loro stato}
}

\newglossaryentry{ideg}{
    name=\glslink{ide}{IDE},
    text=\textit{Integrated Development Environment},
    sort=ide,
    description={Un IDE (Integrated Development Environment) è un software che fornisce un ambiente di sviluppo integrato per la scrittura, il debugging ed il testing di codice. Questo software fornisce agli sviluppatori tutti gli strumenti necessari per scrivere e testare il codice in un unico ambiente}
}

\newglossaryentry{spag}{
    name=\glslink{spa}{SPA},
    text=\textit{Single Page Application},
    sort=spa,
    description={Il termine \textit{Single Page Application} si riferisce ad un'applicazione web che carica una sola pagina HTML e aggiorna dinamicamente il contenuto della pagina senza ricaricare l'intera pagina. Questo approccio permette di creare applicazioni web più veloci e reattive} 
}

\newglossaryentry{prg}{
    name=\glslink{pr}{PR},
    text=\textit{Pull Request},
    sort=pr,
    description={Il termine \textit{Pull Request} si riferisce ad una richiesta di un membro del team di unire un proprio branch (sia esso di feature o bugix) con il codice principale (develop/main) del progetto. Questa richiesta viene utilizzata per revisionare il codice, discutere le modifiche e garantire che il codice sia conforme agli standard del progetto prima di unirlo al codice principale}
}

\newglossaryentry{odmg}{
    name=\glslink{odm}{ODM},
    text=\textit{Object Data Modeling},
    sort=odm,
    description={Il termine \textit{Object Data Modeling} si riferisce al processo di progettazione e creazione di modelli di dati basati su oggetti. Questo approccio permette di rappresentare i dati in modo più naturale e flessibile rispetto ai tradizionali modelli di dati relazionali}
}

\newglossaryentry{llmg}{
    name=\glslink{llm}{LLM},
    text=\textit{Large Language Model},
    sort=llm,
    description={Il termine \textit{Large Language Model} si riferisce ad un modello di linguaggio che utilizza tecniche di apprendimento automatico per generare testo in modo automatico. Questi modelli sono addestrati su grandi quantità di testo per apprendere la struttura e il significato del linguaggio naturale}
}

\newglossaryentry{jwtg}{
    name=\glslink{jwt}{JWT},
    text=\textit{JSON Web Token},
    sort=jwt,
    description={Il termine \textit{JSON Web Token} si riferisce ad uno standard aperto che definisce un modo compatto e autonomo per rappresentare informazioni tra due parti. Questo standard è utilizzato per trasmettere informazioni tra il client e il server in modo sicuro e affidabile}
}

\newglossaryentry{aig}{
    name=\glslink{ai}{AI},
    text=\textit{Articial Intelligence},
    sort=ai,
    description={Il termine \textit{Artificial Intelligence} si riferisce ad un campo dell'informatica che si occupa dello sviluppo di algoritmi e modelli che permettono ai computer di eseguire compiti che richiedono intelligenza umana. Questi algoritmi e modelli sono in grado di apprendere dai dati e di migliorare le proprie prestazioni nel tempo}
}

\newglossaryentry{httpg}{
    name=\glslink{http}{HTTP},
    text=\textit{Hypertext Transfer Protocol},
    sort=http,    
    description={Il termine \textit{Hypertext Transfer Protocol} si riferisce ad un protocollo di comunicazione utilizzato per trasferire informazioni tra il client e il server su Internet. Questo protocollo è basato sul concetto di richiesta e risposta, dove il client invia una richiesta al server e il server risponde con i dati richiesti}
}

\newglossaryentry{ragg}{
    name=\glslink{rag}{RAG},
    text=\textit{Retrieval Augmented Generation},
    sort=rag,    
    description={Il termine \textit{Retrieval Augmented Generation} si riferisce ad un approccio ibrido che combina tecniche di generazione di testo e di recupero di informazioni. Questo approccio permette di generare testo in modo automatico utilizzando informazioni recuperate da una base di conoscenza}
}

\newglossaryentry{frontend}{
    name=\glslink{frontend}{front-end},
    text=\textit{Frontend},
    sort=frontend,
    description={Il termine \textit{front-end} si riferisce alla parte di un'applicazione che interagisce con l'utente. Questa parte dell'applicazione è responsabile della presentazione dei dati all'utente e della gestione delle interazioni con l'utente}
}

\newglossaryentry{backend}{
    name=\glslink{backend}{back-end},
    text=\textit{Backend},
    sort=backend,
    description={Il termine \textit{back-end} si riferisce alla parte di un'applicazione che gestisce i dati e la logica di business dell'applicazione. Questa parte dell'applicazione è responsabile della gestione dei dati, della sicurezza e delle prestazioni dell'applicazione}
}

\newglossaryentry{branch}{
    name=\glslink{branch}{Branch},
    text=\textit{branch},
    sort=branch,
    description={Il termine \textit{branch} si riferisce ad una copia separata del codice sorgente di un progetto. Questa copia separata viene utilizzata per sviluppare nuove funzionalità o correggere bug senza influenzare il codice principale (develop/main) del progetto}
}

\newglossaryentry{machine-learning}{
    name=\glslink{machine-learning}{Machine Learning},
    text=\textit{Machine Learning},
    sort=machine-learning,
    description={Il termine \textit{Machine Learning} si riferisce ad un campo dell'intelligenza artificiale che si occupa dello sviluppo di algoritmi e modelli che permettono ai computer di apprendere dai dati e di migliorare le proprie prestazioni nel tempo}
}

\newglossaryentry{container}{
    name=\glslink{container}{Container},
    text=\textit{container},
    sort=container,
    description={Il termine \textit{container} si riferisce ad un'unità software che contiene un'applicazione e tutte le sue dipendenze. Questa unità software è isolata dal sistema operativo e può essere eseguita su qualsiasi sistema che supporta i container}
}

\newglossaryentry{prompt-engineering}{
    name=\glslink{prompt-engineering}{Prompt Engineering},
    text=\textit{Prompt Engineering},
    sort=prompt-engineering,
    description={Il termine \textit{Prompt Engineering} si riferisce al processo di progettazione e creazione di prompt per un modello di linguaggio. Questo processo è fondamentale per ottenere risultati accurati e coerenti da un modello di linguaggio}
}





