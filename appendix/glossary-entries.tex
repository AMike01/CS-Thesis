% Acronyms
\newacronym[description={\glslink{apig}{Application Program Interface}}]
    {api}{API}{Application Program Interface}

\newacronym[description={\glslink{umlg}{Unified Modeling Language}}]
    {uml}{UML}{Unified Modeling Language}

\newacronym[description={\glslink{llmg}{Large Language Model}}]
    {llm}{LLM}{\textit{Large Language Model}}

% Glossary entries
\newglossaryentry{apig} {
    name=\glslink{api}{API},
    text=Application Program Interface,
    sort=api,
    description={in informatica con il termine \emph{Application Programming Interface API} (ing. interfaccia di programmazione di un'applicazione) si indica ogni insieme di procedure disponibili al programmatore, di solito raggruppate a formare un set di strumenti specifici per l'espletamento di un determinato compito all'interno di un certo programma. La finalità è ottenere un'astrazione, di solito tra l'hardware e il programmatore o tra software a basso e quello ad alto livello semplificando così il lavoro di programmazione}
}

\newglossaryentry{umlg} {
    name=\glslink{uml}{UML},
    text=UML,
    sort=uml,
    description={in ingegneria del software \emph{UML, Unified Modeling Language} (ing. linguaggio di modellazione unificato) è un linguaggio di modellazione e specifica basato sul paradigma object-oriented. L'\emph{UML} svolge un'importantissima funzione di ``lingua franca'' nella comunità della progettazione e programmazione a oggetti. Gran parte della letteratura di settore usa tale linguaggio per descrivere soluzioni analitiche e progettuali in modo sintetico e comprensibile a un vasto pubblico}
}

\newglossaryentry{llmg}{
    name=\glslink{llm}{LLM},
    text=\textit{Large Language Model},
    sort=llm,
    description={Il termine \textit{Large Language Model} si riferisce ad un modello di linguaggio che utilizza tecniche di apprendimento automatico per generare testo in modo automatico. Questi modelli sono addestrati su grandi quantità di testo per apprendere la struttura e il significato del linguaggio naturale}
}

\newglossaryentry{project-manager} {
    name=\glslink{project-manager}{Project Manager},
    text=\textit{Project Manager},
    sort=project manager,
    description={Il \emph{Project Manager} è la figura professionale che si occupa della gestione di un progetto. Egli è responsabile della pianificazione, dell'organizzazione e del controllo delle attività necessarie per il raggiungimento degli obiettivi prefissati}
}

\newglossaryentry{ai-generativa} {
    name=\glslink{ai-generativa}{AI Generativa},
    text=AI Generativa,
    sort=AI Generativa,
    description={L'\emph{AI Generativa} è un campo dell'Intelligenza Artificiale che si occupa di creare nuovi contenuti, come immagini, suoni o testi, a partire da un insieme di dati di partenza}
}

\newglossaryentry{prompt} {
    name=\glslink{prompt}{Prompt},
    text=\textit{Prompt},
    sort=prompt,
    description={Il \emph{Prompt} è una frase o un paragrafo che viene utilizzato per guidare un sistema di Intelligenza Artificiale nella creazione di contenuti}
}

\newglossaryentry{token} {
    name=token,
    text=\textit{token},
    sort=token,
    description={Il termine \emph{token} si riferisce ad una sequenza di caratteri che rappresenta un'unità di informazione. Questi possono essere parole intere, parti di parole, o anche singoli caratteri, a seconda del modello e della sua configurazione}
}