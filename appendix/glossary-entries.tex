% Acronyms
\newacronym[description={\glslink{apig}{Application Program Interface}}]
    {api}{API}{Application Program Interface}

\newacronym[description={\glslink{umlg}{Unified Modeling Language}}]
    {uml}{UML}{Unified Modeling Language}

\newacronym[description={\glslink{llmg}{Large Language Model}}]
    {llm}{LLM}{\textit{Large Language Model}}

\newacronym[description={\glslink{aig}{Artificial Intelligence}}]
    {ai}{AI}{\textit{Artificial Intelligence}}

\newacronym[description={\glslink{awsg}{Amazon Web Services}}]
    {aws}{AWS}{\textit{Amazon Web Services}}    

% Glossary entries
\newglossaryentry{apig} {
    name=\glslink{api}{API},
    text=Application Program Interface,
    sort=api,
    description={in informatica con il termine \emph{Application Programming Interface API} (ing. interfaccia di programmazione di un'applicazione) si indica ogni insieme di procedure disponibili al programmatore, di solito raggruppate a formare un set di strumenti specifici per l'espletamento di un determinato compito all'interno di un certo programma. La finalità è ottenere un'astrazione, di solito tra l'hardware e il programmatore o tra software a basso e quello ad alto livello semplificando così il lavoro di programmazione}
}

\newglossaryentry{umlg} {
    name=\glslink{uml}{UML},
    text=UML,
    sort=uml,
    description={in ingegneria del software \emph{UML, Unified Modeling Language} (ing. linguaggio di modellazione unificato) è un linguaggio di modellazione e specifica basato sul paradigma object-oriented. L'\emph{UML} svolge un'importantissima funzione di ``lingua franca'' nella comunità della progettazione e programmazione a oggetti. Gran parte della letteratura di settore usa tale linguaggio per descrivere soluzioni analitiche e progettuali in modo sintetico e comprensibile a un vasto pubblico}
}

\newglossaryentry{awsg}{
    name=\glslink{aws}{AWS},
    text=\textit{Amazon Web Services},
    sort=aws,
    description={\emph{Amazon Web Services} è una piattaforma di servizi cloud offerta da Amazon. Questi servizi sono divisi in diverse categorie, tra cui calcolo, storage, database, analisi, machine learning, sicurezza, sviluppo di applicazioni, mobile, IoT, IA, AR e VR, media e sviluppo di giochi. AWS offre oltre 200 servizi completi da data center in tutto il mondo}
}

\newglossaryentry{llmg}{
    name=\glslink{llm}{LLM},
    text=\textit{Large Language Model},
    sort=llm,
    description={Il termine \textit{Large Language Model} si riferisce ad un modello di linguaggio che utilizza tecniche di apprendimento automatico per generare testo in modo automatico. Questi modelli sono addestrati su grandi quantità di testo per apprendere la struttura e il significato del linguaggio naturale}
}

\newglossaryentry{project-manager} {
    name=\glslink{project-manager}{Project Manager},
    text=\textit{Project Manager},
    sort=project manager,
    description={Il \emph{Project Manager} è la figura professionale che si occupa della gestione di un progetto. Egli è responsabile della pianificazione, dell'organizzazione e del controllo delle attività necessarie per il raggiungimento degli obiettivi prefissati}
}

\newglossaryentry{aig}{
    name=\glslink{ai}{AI},
    text=\textit{Articial Intelligence},
    sort=ai,
    description={Il termine \textit{Artificial Intelligence} si riferisce ad un campo dell'informatica che si occupa dello sviluppo di algoritmi e modelli che permettono ai computer di eseguire compiti che richiedono intelligenza umana. Questi algoritmi e modelli sono in grado di apprendere dai dati e di migliorare le proprie prestazioni nel tempo}
}

\newglossaryentry{ai-generativa} {
    name=\glslink{ai-generativa}{AI Generativa},
    text=AI Generativa,
    sort=AI Generativa,
    description={L'\emph{AI Generativa} è un campo dell'Intelligenza Artificiale che si occupa di creare nuovi contenuti, come immagini, suoni o testi, a partire da un insieme di dati di partenza}
}

\newglossaryentry{prompt} {
    name=\glslink{prompt}{Prompt},
    text=\textit{Prompt},
    sort=prompt,
    description={Il \emph{Prompt} è una frase o un paragrafo che viene utilizzato per guidare un sistema di Intelligenza Artificiale nella creazione di contenuti}
}

\newglossaryentry{token} {
    name=token,
    text=\textit{token},
    sort=token,
    description={Il termine \emph{token} si riferisce ad una sequenza di caratteri che rappresenta un'unità di informazione. Questi possono essere parole intere, parti di parole, o anche singoli caratteri, a seconda del modello e della sua configurazione}
}

\newglossaryentry{frontend}{
    name=Frontend,
    text=\textit{Frontend},
    sort=Frontend,
    description={Il termine \textit{front-end} si riferisce alla parte di un'applicazione che interagisce con l'utente. Questa parte dell'applicazione è responsabile della presentazione dei dati all'utente e della gestione delle interazioni con l'utente}
}

\newglossaryentry{backend}{
    name=Backend,
    text=\textit{Backend},
    sort=Backend,
    description={Il termine \textit{back-end} si riferisce alla parte di un'applicazione che gestisce i dati e la logica di business dell'applicazione. Questa parte dell'applicazione è responsabile della gestione dei dati, della sicurezza e delle prestazioni dell'applicazione}
}

\newglossaryentry{user-stories}{
    name=User Stories,
    text=\textit{User Stories},
    sort=User Stories,
    description={Le \textit{User stories} rappresentano una pratica \textit{Agile}, che viene utilizzata per comprendere le esigenze dell’utente, mediante una descrizione informale, semplice e concisa della funzionalità}
}

\newglossaryentry{sprint}{
    name=sprint,
    text=\textit{sprint},
    sort=sprint,
    description={Il termine \textit{sprint} si riferisce a un periodo di tempo delimitato, di solito una o due settimane, durante il quale il \textit{team} di sviluppo lavora per completare un insieme di obiettivi prefissati}
}

\newglossaryentry{git-flow}{
    name=Git Flow,
    text=\textit{Git Flow},
    sort=Git Flow,
    description={Il termine \textit{Git Flow} si riferisce a un modello di \textit{branching} per il controllo delle versioni con \textit{Git}. Questo modello prevede l'utilizzo di due branch principali, \textit{master} e \textit{develop}, e di branch di \textit{feature}, \textit{release} e \textit{hotfix} per organizzare il lavoro di sviluppo}
}