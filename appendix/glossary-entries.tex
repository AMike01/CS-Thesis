% Acronyms
\newacronym[description={\glslink{apig}{Application Program Interface}}]
    {api}{API}{Application Program Interface}

\newacronym[description={\glslink{umlg}{Unified Modeling Language}}]
    {uml}{UML}{Unified Modeling Language}

\newacronym[description={\glslink{awsg}{Amazon Web Services}}]
    {aws}{AWS}{Amazon Web Services}

% Glossary entries
\newglossaryentry{apig} {
    name=\glslink{api}{API},
    text=Application Program Interface,
    sort=api,
    description={in informatica con il termine \emph{Application Programming Interface API} (ing. interfaccia di programmazione di un'applicazione) si indica ogni insieme di procedure disponibili al programmatore, di solito raggruppate a formare un set di strumenti specifici per l'espletamento di un determinato compito all'interno di un certo programma. La finalità è ottenere un'astrazione, di solito tra l'hardware e il programmatore o tra software a basso e quello ad alto livello semplificando così il lavoro di programmazione}
}

\newglossaryentry{umlg} {
    name=\glslink{uml}{UML},
    text=UML,
    sort=uml,
    description={in ingegneria del software \emph{UML, Unified Modeling Language} (ing. linguaggio di modellazione unificato) è un linguaggio di modellazione e specifica basato sul paradigma object-oriented. L'\emph{UML} svolge un'importantissima funzione di ``lingua franca'' nella comunità della progettazione e programmazione a oggetti. Gran parte della letteratura di settore usa tale linguaggio per descrivere soluzioni analitiche e progettuali in modo sintetico e comprensibile a un vasto pubblico}
}

\newglossaryentry{awsg}{
    name=\glslink{aws}{AWS},
    text=Amazon Web Services,
    sort=aws,
    description={\emph{Amazon Web Services} è una piattaforma di servizi cloud offerta da Amazon. Questi servizi sono divisi in diverse categorie, tra cui calcolo, storage, database, analisi, machine learning, sicurezza, sviluppo di applicazioni, mobile, IoT, IA, AR e VR, media e sviluppo di giochi. AWS offre oltre 200 servizi completi da data center in tutto il mondo}
}

\newglossaryentry{user-stories}{
    name=User Stories,
    text=User Stories,
    sort=User Stories,
    description={Le \textit{User stories} rappresentano una pratica \textit{Agile}, che viene utilizzata per comprendere le esigenze dell’utente, mediante una descrizione informale, semplice e concisa della funzionalità.}
}

\newglossaryentry{product-backlog}{
    name=Product Backlog,
    text=Product Backlog,
    sort=Product Backlog,
    description={Il \textit{Product Backlog} è una lista di tutte le funzionalità, i requisiti, le modifiche e le correzioni che devono essere fatte ad un prodotto. Questa lista è dinamica e viene aggiornata costantemente, in modo da riflettere le esigenze del cliente.}
}

