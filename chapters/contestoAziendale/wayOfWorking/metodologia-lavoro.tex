\subsection{Metodologia di lavoro}
\label{sez:metodologia-lavoro}

Durante il mio periodo di \textit{stage} ho potuto sperimentare in prima persona la metodologia di lavoro adottata da \textbf{Zero12} per lo sviluppo dei progetti.\\
L'azienda utilizza una metodologia di sviluppo \textit{Agile}, utilizzando il \textit{framework} \textit{Scrum}, che permette di gestire progetti complessi e di portata variabile,
garantendo una maggiore flessibilità e adattabilità ai cambiamenti.\\
Ogni progetto viene diviso in \textit{sprint} di durata variabile, solitamente di una o due settimane, durante le quali vengono sviluppate le funzionalità concordate con il cliente.\\
Durante il mio \textit{stage} la lunghezza di ogni \textit{sprint} è stata di una settimana, per poter avere un \textit{feedback} più rapido sul lavoro svolto.\\
Ogni progetto viene gestito da un \textit{team} composto da un \textit{Product Owner}, uno \textit{Scrum Master} ed uno o più \textit{developer}, ognuno con un ruolo ben definito all'interno del progetto.\\

Il flusso che ogni progetto segue è il seguente:
\begin{itemize}
    \item \textbf{\textit{\gls{user-stories}}:} il \textit{Product Owner} definisce le funzionalità richieste dal cliente sotto forma di \textit{user stories}.\\
    Le \gls{user-stories} sono brevi descrizioni delle funzionalità richieste dal cliente, scritte in modo da essere comprensibili sia per il cliente che per il \textit{team} di sviluppo.
    Queste saranno poi inserite all'interno del \gls{product-backlog}, ovvero la lista delle attività da svolgere all'interno del progetto.
    \item \textbf{\textit{Sprint Planning}:} all'inizio di ogni \textit{sprint} il \textit{team} si riunisce per pianificare le attività da svolgere durante la settimana, selezionando le \gls{user-stories} da sviluppare, 
    in base alla loro priorità ed alla loro complessità.
    \item \textbf{\textit{Daily Standup}:} ogni giorno il \textit{team} si riunisce per fare il punto della situazione, ognuno dei membri del \textit{team} riporta come procede il lavoro,
    cosa è stato fatto il giorno precedente e cosa si prevede di fare durante la giornata.
    \item \textbf{\textit{Sprint Review}:} alla fine di ogni \textit{sprint} il \textit{team} si riunisce con il cliente per mostrare le funzionalità sviluppate durante la settimana, assicurandosi che il lavoro sia stato svolto efficientemente e 
    che soddisfi le aspettative del cliente.
    \item \textbf{\textit{Sprint Retrospective}:} alla fine di ogni \textit{sprint} il \textit{team} si riunisce per fare il punto della situazione dello \textit{sprint} appena terminato, 
    valutando il lavoro svolto e cercando miglioramenti per i prossimi \textit{sprint}.
\end{itemize}
