\pagebreak
\section{\textit{Large Language Models}}
\label{sez:llm}

\subsection{Introduzione}
\label{subsec:llm-introduzione}

I \gls{llmg} sono modelli computazionali progettati per interpretare e generare testo in linguaggio naturale\footcite{site:introduction-llm}.\\
Sebbene le loro basi teoriche risalgano alla ``Teoria dell'apprendimento statistico'' sviluppata tra gli anni '80 e '90, è solo negli ultimi anni che hanno guadagnato un'attenzione globale significativa. \\
Questo è stato possibile grazie alla crescente disponibilità di dati e di potenza di calcolo, che hanno permesso la creazione di modelli sempre più grandi e complessi.\\
Le applicazioni dei \gls{llmg} sono molteplici: traduzioni automatiche, generazione di testi, riconoscimento vocale, e \textit{chatbot} sono solo alcuni esempi.\\

\noindent Il principio di funzionamento di un \gls{llmg} si basa sull'addestramento su vastissime quantità di dati testuali, come libri, articoli \textit{web} e altre risorse scritte, 
per apprendere la struttura e il significato del linguaggio naturale.\\
Successivamente, questi modelli sono in grado di generare testo coerente e sensato in base al contesto fornito.\\

\noindent I \gls{llmg} utilizzano un approccio ``probabilistico'', ovvero cercano di prevedere la parola successiva in una sequenza testuale basandosi sulle parole precedenti.\\
Ogni parola ha una probabilità associata, calcolata dal modello. \\
La parola con la probabilità più alta viene spesso selezionata per garantire coerenza, ma ciò può portare a risultati prevedibili e poco creativi, che possono risultare poco interessanti per l'utente.\\

\noindent Per ovviare a questo problema, si utilizzano i seguenti parametri\footcite{site:mediun-llm-parameters}:
\begin{itemize}
    \item \textbf{Temperatura}: regola la casualità delle scelte del modello. \\
    Un valore basso predilige la scelta della parola con probabilità più alta, mentre un valore alto aumenta la casualità delle scelte.
    \item \textbf{\textit{Top-K}}: limita la scelta alle \textit{top-K} parole con probabilità più alta.\\
    Questo permette di evitare scelte troppo casuali e poco coerenti, andando a bilanciare prevedibilità e casualità.\\
\end{itemize}

\noindent Mostriamo ora un esempio: supponiamo che una \gls{llmg} stia generando una frase a partire dalla seguente sequenza di parole: 
\begin{center}
    \textit{``Il sistema ha ricevuto una richiesta da''}\\
\end{center}

Le probabilità per le parole successive potrebbero essere le seguenti:
\begin{itemize}
    \item \textit{``un utente autenticato''}: 0.4
    \item \textit{``un client esterno''}: 0.3
    \item \textit{``un microservizio interno''}: 0.2
    \item \textit{``un server remoto''}: 0.1
\end{itemize}
\pagebreak
\begin{enumerate}
    \item Senza utilizzo dei parametri, il modello seleziona la parola con probabilità più alta: \\
    \textit{``Il sistema ha ricevuto una richiesta da \textbf{un utente autenticato}''};
    \item Con temperatura alta (ad esempio 1.0), il modello seleziona una parola in modo casuale, ad esempio: \\
    \textit{``Il sistema ha ricevuto una richiesta da \textbf{un server remoto}''};
    \item Con \textit{top-K} basso (ad esempio 2), il modello seleziona una parola tra le due con probabilità più alta, ad esempio: \\
    \textit{``Il sistema ha ricevuto una richiesta da \textbf{un client esterno}''}.
\end{enumerate}

\noindent Un altro concetto fondamentale per comprendere il funzionamento dei \gls{llmg} è quello dei \gls{token}.\\
Un \gls{token} rappresenta la più piccola unità di testo che un \gls{llm} può elaborare.\\
I \gls{token} possono essere parole intere, parti di parole o anche singoli caratteri, a seconda del modello e della sua configurazione.\\

\noindent La suddivisione in \gls{token} è cruciale per l'addestramento dei \gls{llmg}, poiché il modello deve essere in grado di gestire e prevedere sequenze di \gls{token} piuttosto che intere frasi o parole.\\
Durante l'addestramento, il modello apprende le probabilità di transizione tra i \gls{token}, costruendo una rappresentazione statistica del linguaggio naturale.\\



