\subsection{Confronto tra principali LLM}
\label{subsec:llm-confronto}

Di seguito è presente un confronto tra i principali \gls{llm} attualmente disponibili sul mercato\footcite{site:key-differences-llm}, con l'obiettivo di evidenziare le loro differenze, i casi d'uso principali, i loro punti di forza ed i limiti di ciascun modello.\\
Questi modelli sono accessibili tramite \textit{AWS Bedrock}\footcite{site:aws-bedrock}, ad eccezione di \textit{GPT-4}, incluso comunque nel confronto per la sua rilevanza e diffusione.

\subsubsection{\textit{GPT-4 (OpenAI)}}

\noindent \textbf{Descrizione:}
\textit{GPT-4} è un modello all’avanguardia progettato per risolvere compiti complessi grazie alla sua elevata capacità di comprensione e generazione del linguaggio. \\
Anche se non disponibile tramite \textit{AWS Bedrock}, è una delle soluzioni più utilizzate per la sua accuratezza e versatilità.\\

\noindent \textbf{Casi d’uso:}
\begin{itemize}
   \item \textbf{Generazione tecnica:} Creazione di documentazione tecnica dettagliata;
   \item \textbf{Analisi avanzata:} Interpretazione di dati e contenuti complessi;
   \item \textbf{Supporto per sviluppatori:} \textit{Debugging} e generazione di codice ottimizzato;
   \item \textbf{Ricerca scientifica:} Sintesi e analisi di \textit{paper} scientifici.
\end{itemize}

\noindent \textbf{Punti di forza:}
\begin{itemize}
    \item \textbf{Capacità di contesto estesa:} Supporta fino a 32k \gls{token}, adatto per documenti complessi;
    \item \textbf{Versatilità \textit{multi-task}:} Funziona bene su domande aperte e contesti tecnici specifici;
    \item \textbf{\textit{Fine-tuning} implicito:} Non necessita di \textit{tuning} per molti casi d’uso, grazie ad un addestramento generale molto efficace.
\end{itemize}

\noindent \textbf{Limiti:}
\begin{itemize}
    \item \textbf{Costo elevato:} Il prezzo per \gls{token} è più alto rispetto ad altre opzioni;
    \item \textbf{Latenza:} \textit{Input} complessi possono rallentare il tempo di risposta.
    \item \textbf{Non nativo per \textit{AWS Bedrock}:} Necessita di integrazione tramite \gls{api} \textit{OpenAI}.
\end{itemize}

\vspace{1.5cm}
\subsubsection{\textit{Claude 3.5 Sonnet (Anthropic)}}

\noindent \textbf{Descrizione:}
\textit{Claude 3.5 Sonnet} è ottimizzato per la gestione di contesti estesi e la generazione di risposte sicure e trasparenti. \\\
Su \textit{AWS Bedrock}, è utilizzato per applicazioni con grandi volumi di testo e analisi linguistiche. \\

\noindent \textbf{Casi d’uso:}
\begin{itemize}
    \item \textbf{Analisi documentale:} Generazione e sintesi di documenti di grandi dimensioni (fino a 100k \gls{token});
    \item \textbf{Automazione \textit{HR} e legale:} Risposte standardizzate e conformità normativa;
    \item \textbf{Interfacce conversazionali:} \textit{Chatbot} sicuri per il supporto clienti ed interno.
\end{itemize}

\noindent \textbf{Punti di forza:}
\begin{itemize}
    \item \textbf{Capacità di contesto estesa:} Supporta \textit{input} fino a 100k \gls{token}, ideale per documenti lunghi;
    \item \textbf{Bias ridotto:} Allenato per minimizzare risposte non sicure o inappropriate;
    \item \textbf{Costi contenuti:} Efficiente rapporto costo\textit{-token}.
\end{itemize}

\noindent \textbf{Limiti:}
\begin{itemize}
    \item \textbf{Latenza:} \textit{Input} complessi possono rallentare il tempo di risposta;
    \item \textbf{Risposte conservative:} Tende a evitare \textit{output} creativi o fuori dall’ordinario.
    \item \textbf{Limitato per codice:} Meno supporto per generazione di codice avanzato.
\end{itemize}

\subsubsection{\textit{Amazon Titan}}

\noindent \textbf{Descrizione:}
\textit{Titan} è il modello di intelligenza artificiale sviluppato da \textit{Amazon}, ottimizzato per l’integrazione nativa con \textit{AWS Bedrock} ed altre soluzioni \gls{aws}.\\
Si concentra su casi d’uso aziendali pratici e robusti.\\

\noindent \textbf{Casi d’uso:}
\begin{itemize}
    \item \textbf{Elaborazione dati strutturati:} Supporto per analisi di \textit{report} aziendali e \textit{database};
    \item \textbf{\textit{Chatbot} aziendali:} Personalizzabili per specifici settori industriali;
    \item \textbf{Supporto operativo:} Automazione di flussi di lavoro e riepiloghi aziendali.
\end{itemize}

\noindent \textbf{Punti di forza:}
\begin{itemize}
    \item \textbf{Integrazione nativa con \gls{aws}:} Compatibilità totale con strumenti \gls{aws} come \textit{DynamoDB} e \textit{S3};
    \item \textbf{Ottimizzazione della sicurezza:} Protezione dei dati aziendali con politiche \gls{aws} rigorose;
    \item \textbf{Efficienza nei costi:} Disegnato per fornire prestazioni elevate anche con \textit{budget} contenuti.
\end{itemize}

\noindent \textbf{Limiti:}
\begin{itemize}
    \item \textbf{Capacità di linguaggio generico inferiore:} Non raggiunge la qualità linguistica di \textit{GPT-4} o \textit{Claude 3.5};
    \item \textbf{Meno innovativo:} Adatto a compiti definiti, non eccelle in creatività o flessibilità;
    \item \textbf{Limitata flessibilità:} Richiede configurazioni specifiche per settori diversi.
\end{itemize}

\vspace{1.5cm}
\subsubsection{\textit{Llama (Meta)}}

\noindent \textbf{Descrizione:}
\textit{Llama} è un modello \textit{open-source} progettato per essere altamente personalizzabile, ideale per casi in cui è richiesto un controllo completo sul \textit{fine-tuning}. 
Su \textit{AWS Bedrock}, è particolarmente utile per applicazioni che richiedono soluzioni specifiche per il dominio applicativo.\\

\noindent \textbf{Casi d’uso:}
\begin{itemize}
    \item \textbf{Addestramento personalizzato:} Modelli ottimizzati per domini ristretti come finanza o medicina;
    \item \textbf{Soluzioni locali:} Applicazioni in ambienti chiusi per garantire massima \textit{privacy};
    \item \textbf{Sperimentazione:} Perfetto per ricerca accademica e/o sviluppo di prototipi.
\end{itemize}

\noindent \textbf{Punti di forza:}
\begin{itemize}
    \item \textbf{\textit{Open-source}:} Offre pieno controllo per modifiche e ottimizzazioni;
    \item \textbf{Scalabilità flessibile:} Può essere ridimensionato per applicazioni \textit{on-premises} o \textit{cloud}.
\end{itemize}

\noindent \textbf{Limiti:}
\begin{itemize}
    \item \textbf{Richiede competenze tecniche avanzate:} Il \textit{tuning} e l’implementazione necessitano di risorse specializzate;
    \item \textbf{Prestazioni inferiori senza \textit{tuning}:} Il modello base è meno performante rispetto a \textit{GPT-4} o \textit{Claude 3.5};
    \item \textbf{Ambiente di sviluppo più complesso:} Meno pronto all’uso rispetto ai modelli pre-addestrati.
\end{itemize}

\noindent Di seguito una tabella che mostra un confronto riassuntivo tra i modelli presentati:

\begin{table}[H]
    \centering
    \rowcolors{2}{gray!15}{white} % Righe alternate: grigio chiaro e bianco
    \begin{tabularx}{\textwidth}{|>{\centering\arraybackslash}X|>{\centering\arraybackslash}X|>{\centering\arraybackslash}X|>{\centering\arraybackslash}X|}
        \hline
        \rowcolor{green!30} % Colore dell'intestazione: verde chiaro
        \textbf{Modello} & \textbf{Casi d'uso} & \textbf{Punti di forza} & \textbf{Limiti} \\
        \hline
        \textit{GPT-4} & Automazione avanzata, analisi tecnica & Supporta 32k \textit{token}, versatile e accurato & Costoso, non nativo su \textit{AWS Bedrock} \\
        \hline
        \textit{Claude 3.5} & Generazione documenti lunghi, chatbot sicuri & Contesto ampio, sicurezza nei dati & Risposte conservative, latenza elevata \\
        \hline
        \textit{Amazon Titan} & Report aziendali, automazione & Integrazione nativa \gls{aws}, costi ridotti & Meno versatile, output linguistico generico \\
        \hline
        \textit{Llama} & Addestramento personalizzato & \textit{Open-source}, flessibilità & Prestazioni limitate senza modifiche avanzate \\
        \hline
    \end{tabularx}
    \caption{Tabella confronto riassuntivo \textit{LLM}}
    \label{tab:confronto-lllm}
\end{table}
