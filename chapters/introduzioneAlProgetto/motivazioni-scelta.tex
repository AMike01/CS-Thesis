\section{Obiettivi personali}
\label{sez:obiettivi-personali}

La scelta di intraprendere questo progetto di \textit{stage} è stata motivata da diversi fattori che ritengo molto rilevanti per la mia crescita professionale e per il mio interesse verso tematiche innovative e tecnologiche.\\

\noindent Di seguito sono elencate le motivazioni principali:

\begin{itemize}
    \item \textbf{Tema del progetto:} L'\gls{generative-ai} è un campo estremamente affascinante e in continua evoluzione, con applicazioni che spaziano in molti settori.\\
    L'opportunità di lavorare su un progetto che utilizza queste tecnologie rappresenta una sfida stimolante ed un'occasione unica per approfondire le mie conoscenze in un contesto pratico;
    \item \textbf{Prima impressione a \textit{STAGE-IT}:} Il colloquio tenuto con i rappresentanti aziendali a \textit{STAGE-IT}, un evento tenuto ad Aprile 2024 che permette alle aziende di presentare i propri progetti di \textit{stage}, è risultato molto interessante. \\
    L'incontro con alcuni dei dipendenti aziendali mi ha fatto una buonissima impressione, si sono dimostrati molto disponibili nel rispondere alle mie domande ed ho trovato stimolante il loro entusiasmo nell'affrontare tematiche tecnologiche avanzate;
    \item \textbf{Utilizzo di servizi \gls{aws}:} Sono molto interessato ai servizi \gls{aws}, che considero fondamentali nel contesto lavorativo odierno. \\
    \gls{aws} offre una vasta gamma di servizi \textit{cloud}, essenziali per lo sviluppo di applicazioni moderne e scalabili, e credo che acquisire esperienza con questi strumenti possa essere un vantaggio significativo per la mia carriera lavorativa.\\
    Essendo Zero12 un partner \gls{aws}, questo progetto mi offre l'opportunità di lavorare con questi servizi in un contesto pratico e reale;
    \item \textbf{Tecnologie all'avanguardia:} Rispetto ad altre aziende con cui ho avuto colloqui, questa si distingue per l'adozione di tecnologie moderne e innovative. \\
    L'opportunità di lavorare con tecnologie all'avanguardia rappresenta una grande opportunità di apprendimento;
    \item \textbf{Posizione geografica:} L'azienda si trova a circa 40 minuti dalla mia residenza, in una posizione comoda e ben collegata, che rende semplice e pratico il tragitto quotidiano;
    \item \textbf{\textit{Team} giovane e dinamico:} Dopo aver parlato con alcuni studenti che avevano già svolto lo \textit{stage} in azienda, mi è stato riferito che l'ambiente di lavoro è molto positivo.\\
    Il \textit{team} giovane e dinamico è altamente collaborativo e orientato alla crescita professionale, creando un contesto ideale per lo sviluppo delle competenze.
\end{itemize}


\noindent Gli obiettivi personali rappresentano le competenze e le conoscenze che mi sono imposto di acquisire durante il mio periodo di \textit{stage}.\\

\noindent Questi obiettivi sono legati al miglioramento e allo sviluppo delle proprie capacità professionali, sia dal punto di vista tecnico che umano.
La loro definizione aiuta a focalizzarsi su aree specifiche di crescita e a pianificare il percorso formativo in modo mirato. \\

\noindent Nel contesto del mio stage, gli obiettivi personali possono essere suddivisi in tre categorie principali:

\begin{itemize}
    \item \textbf{Obiettivi tecnici}: Si concentrano sull'acquisizione di nuove competenze tecniche, come l'apprendimento di linguaggi di programmazione, l'utilizzo di \textit{framework}, e l'esplorazione di tecnologie avanzate.
    \item \textbf{Obiettivi di crescita personale}: Riguardano lo sviluppo delle \textit{soft skills}, come la gestione del tempo, la comunicazione professionale e la comprensione del contesto aziendale.
    \item \textbf{Obiettivi di autonomia e collaborazione}: Focalizzati sul miglioramento della capacità di lavorare autonomamente e di collaborare efficacemente con i colleghi.
\end{itemize}

\noindent All'interno della seguente tabella sono riportati gli obiettivi personali, aventi la seguente nomenclatura:

\begin{center}
    \textbf{OP}-\textbf{N}
\end{center}

Dove:
\begin{itemize}
    \item OP sta per Obiettivo Personale; \\
    \item $\textbf{N} \in N$ rappresentante il numero progressivo dell'obiettivo.
\end{itemize}

\pagebreak
\renewcommand{\arraystretch}{1.5} % Increases the row height for better vertical space

\begin{longtable}{|c|>{\centering\arraybackslash}p{0.7\textwidth}|} % Adjust the column width with p{0.7\textwidth}
    \hline
    \rowcolor{green!30} % Header color: light green
    \textbf{Codice Obiettivo} & \textbf{Descrizione Obiettivo} \\
    \hline
    \endfirsthead % Start of the table, first header
    
    \hline
    \rowcolor{green!30} % Header color on subsequent pages
    \textbf{Codice Obiettivo} & \textbf{Descrizione Obiettivo} \\
    \hline
    \endhead % Continuation of the table (repeats on every page)
    
    \hline
    \multicolumn{2}{|c|}{\rowcolor{green!30} \textbf{Obiettivi Personali}} \\
    \hline
    \textbf{OP-1} & \textbf{Padroneggiare nuovi linguaggi e \textit{framework}:} Acquisire competenze avanzate nello sviluppo di applicazioni utilizzando \textit{React} per il \gls{frontend} e \textit{NestJS} per il \gls{backend}, implementando progetti reali che sfruttano queste tecnologie. \\
    \hline
    \textbf{OP-2} & \textbf{Esplorare i servizi \textit{cloud} di \gls{aws}:} Imparare ad utilizzare servizi come \textit{AWS Amplify}, \textit{AWS Cognito}, \textit{AWS S3}, e \textit{AWS Bedrock}, comprendendo come integrarli in un’architettura scalabile e moderna. \\
    \hline
    \textbf{OP-3} & \textbf{Competenze in \gls{generative-ai}:} Sviluppare un solido \textit{know-how} nell’utilizzo di tecnologie di \gls{generative-ai}, come l’integrazione di modelli \textit{AI (Claude, GPT)} in progetti pratici. \\
    \hline
    \multicolumn{2}{|c|}{\rowcolor{green!30} \textbf{Obiettivi di Crescita Personale}} \\
    \hline
    \textbf{OP-4} & \textbf{Comprendere il settore professionale:} Ottenere una visione del contesto aziendale del settore tecnologico, analizzando flussi di lavoro e \textit{trend} di mercato, per orientare al meglio il percorso professionale futuro. \\
    \hline
    \textbf{OP-5} & \textbf{Ottimizzare la gestione del tempo:} Sviluppare un approccio strutturato al lavoro, utilizzando strumenti di produttività e tecniche di prioritizzazione per rispettare scadenze e migliorare l’efficienza personale. \\
    \hline
    \textbf{OP-6} & \textbf{Comunicazione professionale efficace:} Rafforzare le capacità di comunicazione scritta e orale per facilitare il dialogo e le collaborazioni. \\
    \hline
    \multicolumn{2}{|c|}{\rowcolor{green!30} \textbf{Obiettivi di Autonomia e Collaborazione}} \\
    \hline
    \textbf{OP-7} & \textbf{Lavoro indipendente:} Incrementare la capacità di gestire \textit{task} e progetti in modo autonomo, prendendo decisioni informate e risolvendo problemi complessi senza supervisione diretta. \\
    \hline
    \textbf{OP-8} & \textbf{Collaborazione proattiva:} Contribuire attivamente al lavoro di squadra, partecipando a \textit{meeting}, condividendo idee e accogliendo \textit{feedback} per migliorare continuamente le proprie performance. \\
    \hline
    \caption{Obiettivi personali dello \textit{stage}} % Table caption
    \label{tab:obiettivi-personali-stage} % Label for referencing the table
\end{longtable}
