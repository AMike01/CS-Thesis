\section{Crescita professionale}
\label{sez:crescita-professionale}

\input{chapters/retrospettivaFinale/difficoltà-incontrate.tex}

\pagebreak
\subsection{Competenze acquisite}
Nel corso del progetto ho sviluppato e perfezionato diverse competenze, che possono essere suddivise in tre categorie principali:

\begin{itemize}
\item \textbf{Competenze tecniche}:
Ho approfondito e migliorato significativamente la mia conoscenza di \textit{TypeScript}, applicandolo con successo in un progetto complesso nel contesto dello sviluppo \textit{web}. \\
Ho inoltre acquisito competenze nell'utilizzo di diversi servizi \gls{aws}, con un focus particolare su \textit{AWS Bedrock}, dedicato alla \gls{generative-ai}.\\
Questo mi ha permesso di affrontare aspetti tecnici avanzati e di integrare nuove tecnologie nel progetto in maniera efficace;
\item \textbf{Competenze metodologiche}:  
Ho avuto modo di lavorare seguendo un approccio \textit{agile}, partecipando attivamente agli \gls{sprint} e rispettandone le tempistiche. \\  
Inoltre, ho affinato la mia capacità di organizzazione autonoma, definendo con precisione le priorità e gestendo le attività necessarie per portare a termine il lavoro in modo efficiente;

\item \textbf{Competenze personali}:  
Grazie alla partecipazione alle \textit{sprint review} e alla presentazione aziendale del progetto, ho migliorato le mie competenze comunicative, imparando a esporre il mio lavoro in maniera chiara, strutturata e professionale. \\  
Ho anche sviluppato ulteriormente le mie capacità di \textit{problem solving}, trovando soluzioni creative e pratiche alle difficoltà incontrate durante le diverse attività del progetto.  
\end{itemize}