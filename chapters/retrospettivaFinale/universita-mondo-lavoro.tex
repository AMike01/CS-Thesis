\section{Università e mondo del lavoro}
\label{sez:universita-mondo-lavoro}

L’esperienza di \textit{stage} mi ha permesso di comprendere in maniera più profonda come università e mondo del lavoro siano strettamente collegati, rappresentando due facce della stessa medaglia.\\
Da un lato, l’università offre una formazione teorica solida e strutturata, che è indispensabile per affrontare qualsiasi contesto professionale. \\
Dall’altro lato, è nel mondo del lavoro che queste conoscenze trovano una vera applicazione, consentendo di affinare, consolidare e trasformare le basi teoriche in competenze pratiche e operative.  \\

\noindent In questo contesto, il corso di ``Ingegneria del \textit{software}'' ha giocato un ruolo fondamentale nel mio percorso.\\
Durante il corso ho avuto l’opportunità di apprendere concetti cruciali per lo sviluppo e la gestione di progetti \textit{software}, come la progettazione, la gestione dei requisiti e l’organizzazione del lavoro in \textit{team}. \\
Tuttavia, ho compreso di non aver sfruttato appieno questa opportunità, a causa di scelte errate effettuate durante l’esecuzione del progetto pratico associato al corso.\\
Tali scelte, dettate forse da inesperienza o da una sottovalutazione dell’importanza di alcuni aspetti del corso, mi hanno impedito di trarre tutto il valore che avrei potuto ottenere. \\ 

\noindent Lo \textit{stage}, però, mi ha dato l’occasione di rimediare a queste mancanze, offrendomi uno spazio in cui applicare e migliorare concretamente le conoscenze acquisite all'interno del corso sopracitato.\\
Ritengo quindi che l'università mi abbia dato delle buonissime basi teoriche, ma è stato solo grazie all'esperienza di \textit{stage} che ho potuto trasformare queste basi in competenze pratiche e operative, fondamentali per affrontare il mondo del lavoro.\\