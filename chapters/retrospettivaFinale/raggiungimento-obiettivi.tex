\section{Raggiungimento degli obiettivi}
\label{sez:raggiungimento-obiettivi}

\subsection{Obiettivi aziendali}
\label{subsec:raggiungimento-obiettivi-aziendali}

Riporto gli obiettivi aziendali indicati nella \hyperref[sez:obiettivi-aziendali]{Sezione §2.4} e ne descrivo il grado di soddisfacimento conseguito a fine \textit{stage}.\\

\renewcommand{\arraystretch}{1.5} % Increases the row height for better vertical space

\begin{longtable}{|c|>{\centering\arraybackslash}p{0.7\textwidth}|c|} % Adjust the column width with p{0.7\textwidth}
    \hline
    \rowcolor{green!30} % Header color: light green
    \textbf{Codice Obiettivo} & \textbf{Descrizione Obiettivo} & \textbf{Soddisfatto}\\
    \hline
    \endfirsthead % Start of the table, first header
    
    \hline
    \rowcolor{green!30} % Header color on subsequent pages
    \textbf{Codice Obiettivo} & \textbf{Descrizione Obiettivo} & \textbf{Soddisfatto}\\
    \hline
    \endhead % Continuation of the table (repeats on every page)
    
    \hline
    \multicolumn{3}{|c|}{\rowcolor{green!30} \textbf{Obiettivi Obbligatori}}  \\
    \hline % Adds a horizontal line after the header row
    \textbf{OO-1} & \textbf{Apprendimento delle tecnologie di sviluppo:} Acquisire competenze pratiche nell'uso di \textit{React}, \textit{NestJS} e \textit{MongoDB} per la progettazione e lo sviluppo di applicazioni. & Sì \\

    \hline
    \textbf{OO-2} & \textbf{Gestione del versionamento del codice:} Apprendere l'uso di \textit{Git} e adottare \textit{Git Flow} come metodologia per il controllo delle versioni e la collaborazione. & Sì\\
    \hline
    \textbf{OO-3} & \textbf{Analisi e scelta del \gls{llm}:} Valutare i modelli disponibili per selezionare quello più adatto al progetto.& Sì \\
    \hline
    \textbf{OO-4} & \textbf{Introduzione alle metodologie agili:} Familiarizzare con le metodologie agili di sviluppo per la gestione efficace di progetti. & Sì\\
    \hline
    \textbf{OO-5} & \textbf{Pianificazione e gestione giornaliera:} Imparare a gestire \textit{task} e obiettivi giornalieri allineati al piano di lavoro. & Sì\\
    \hline
    \textbf{OO-6} & \textbf{Sviluppo di una \textit{web app:}} Progettare e realizzare un'applicazione \textit{web} per consentire l'interazione dell'utente con la piattaforma. & Sì\\
    \hline
    \textbf{OO-7} & \textbf{Sviluppo ed integrazione con \gls{generative-ai}:} Implementare i flussi logici del progetto e integrare i servizi di \gls{generative-ai} scelti. & Sì\\
    \hline
    \textbf{OO-8} & \textbf{Documentazione \gls{api}:} Creare una documentazione \textit{Swagger} per le \gls{api} sviluppate.& Sì \\
    \hline
    \textbf{OO-9} & \textbf{Documento di analisi progettuale:} Redigere un documento tecnico che descriva l'architettura e le componenti principali della piattaforma.& Sì \\
    \hline
    \textbf{OO-10} & \textbf{\textit{User Story Mapping:}} Realizzare una mappatura delle \gls{user-stories} per descrivere e organizzare i requisiti del progetto.& Sì \\
    \hline
    \multicolumn{3}{|c|}{\rowcolor{green!30} \textbf{Obiettivi Desiderabili}} \\
    \hline % Adds a horizontal line after the subtitle
    \textbf{OD-1} & \textbf{Comparazione tra modelli \gls{llm}:} Effettuare un'analisi comparativa tra almeno due \gls{llm} per verificarne le differenze in termini di prestazioni, funzionalità e costi.& No \\
    \hline
    \multicolumn{3}{|c|}{\rowcolor{green!30} \textbf{Obiettivi Facoltativi}} \\
    \hline % Adds a horizontal line after the subtitle
    \textbf{OF-1} & \textbf{Piattaforma di amministrazione:} Creare una piattaforma \textit{admin} per la gestione dei \textit{Preset} e dei modelli \gls{llm}.& No \\
    \hline
    \caption{Obiettivi aziendali dello \textit{stage}} % Table caption
    \label{tab:raggiungimento_obiettivi_stage} % Label for referencing the table
\end{longtable}

\noindent Sono riuscito a soddisfare il 100\% degli obiettivi obbligatori del progetto, garantendo il raggiungimento di tutte le funzionalità essenziali. \\
Tuttavia, non sono riuscito a completare alcuno degli obiettivi desiderabili o facoltativi, a causa di limitazioni tecniche e di tempo.\\

\noindent In particolare, non ho raggiunto l'obiettivo OD-1 (Comparazione tra modelli \gls{llm}) a causa della complessità tecnica nell'integrare diversi modelli con \textit{AWS Bedrock} e nell'effettuare un'analisi approfondita delle loro prestazioni.\\
Inoltre, il tempo a disposizione non è stato sufficiente per completare le configurazioni necessarie.\\

\noindent Allo stesso modo, non sono riuscito a completare lo sviluppo della piattaforma di amministrazione (OF-1).
Questo obiettivo, essendo secondario rispetto alle priorità principali del progetto, non ha potuto ricevere l’attenzione necessaria entro i tempi disponibili.
\pagebreak
\subsection{Obiettivi personali}
\label{subsec:raggiungimento-obiettivi-personali}

Riporto gli obiettivi personali indicati nella \hyperref[sez:obiettivi-personali]{Sezione §2.6} e ne descrivo il grado di soddisfacimento conseguito a fine \textit{stage}.\\

\renewcommand{\arraystretch}{1.5} % Increases the row height for better vertical space

\begin{longtable}{|c|>{\centering\arraybackslash}p{0.7\textwidth}|c|} % Adjust the column width with p{0.7\textwidth}
    \hline
    \rowcolor{green!30} % Header color: light green
    \textbf{Codice Obiettivo} & \textbf{Descrizione Obiettivo} & \textbf{Soddisfatto}\\
    \hline
    \endfirsthead % Start of the table, first header
    
    \hline
    \rowcolor{green!30} % Header color on subsequent pages
    \textbf{Codice Obiettivo} & \textbf{Descrizione Obiettivo}  & \textbf{Soddisfatto}\\
    \hline
    \endhead % Continuation of the table (repeats on every page)
    
    \hline
    \multicolumn{3}{|c|}{\rowcolor{green!30} \textbf{Obiettivi Personali}} \\
    \hline
    \textbf{OP-1} & \textbf{Padroneggiare nuovi linguaggi e \textit{framework}:} Acquisire competenze avanzate nello sviluppo di applicazioni utilizzando \textit{React} per il \gls{frontend} e \textit{NestJS} per il \gls{backend}, implementando progetti reali che sfruttano queste tecnologie. & Sì \\
    \hline
    \textbf{OP-2} & \textbf{Esplorare i servizi \textit{cloud} di \gls{aws}:} Imparare ad utilizzare servizi come \textit{AWS Amplify}, \textit{AWS Cognito}, \textit{AWS S3}, e \textit{AWS Bedrock}, comprendendo come integrarli in un’architettura scalabile e moderna. & Sì\\
    \hline
    \textbf{OP-3} & \textbf{Competenze in \gls{generative-ai}:} Sviluppare un solido \textit{know-how} nell’utilizzo di tecnologie di \gls{generative-ai}, come l’integrazione di modelli \textit{AI (Claude, GPT)} in progetti pratici.& Sì \\
    \hline
    \multicolumn{3}{|c|}{\rowcolor{green!30} \textbf{Obiettivi di Crescita Personale}} \\
    \hline
    \textbf{OP-4} & \textbf{Comprendere il settore professionale:} Ottenere una visione del contesto aziendale del settore tecnologico, analizzando flussi di lavoro e \textit{trend} di mercato, per orientare al meglio il percorso professionale futuro.& Sì \\
    \hline
    \textbf{OP-5} & \textbf{Ottimizzare la gestione del tempo:} Sviluppare un approccio strutturato al lavoro, utilizzando strumenti di produttività e tecniche di prioritizzazione per rispettare scadenze e migliorare l’efficienza personale. & Sì\\
    \hline
    \textbf{OP-6} & \textbf{Comunicazione professionale efficace:} Rafforzare le capacità di comunicazione scritta e orale per facilitare il dialogo e le collaborazioni.& Sì \\
    \hline
    \multicolumn{3}{|c|}{\rowcolor{green!30} \textbf{Obiettivi di Autonomia e Collaborazione}} \\
    \hline
    \textbf{OP-7} & \textbf{Lavoro indipendente:} Incrementare la capacità di gestire \textit{task} e progetti in modo autonomo, prendendo decisioni informate e risolvendo problemi complessi senza supervisione diretta.& Sì \\
    \hline
    \textbf{OP-8} & \textbf{Collaborazione proattiva:} Contribuire attivamente al lavoro di squadra, partecipando a \textit{meeting}, condividendo idee e accogliendo \textit{feedback} per migliorare continuamente le proprie performance. & Parzialmente\\
    \hline
    \caption{Obiettivi personali dello \textit{stage}} % Table caption
    \label{tab:raggiungimento-obiettivi-personali-stage} % Label for referencing the table
\end{longtable}

\noindent Ho raggiunto tutti i miei obiettivi personali, ad eccezione dell'obiettivo OP-8 (Collaborazione proattiva), che sono riuscito a soddisfare solo in parte.\\

\noindent In particolare, il mio obiettivo di contribuire attivamente al lavoro di squadra, partecipando a \textit{meeting}, condividendo idee e accogliendo \textit{feedback}, non è stato pienamente raggiunto.\\
Anche se ho preso parte a diversi \textit{meeting}, ho lavorato principalmente in autonomia e non mi sono state fornite opportunità di collaborare attivamente con un \textit{team}.