\vspace{0.5cm}
\section*{\textit{Epic} 1: Accesso alla piattaforma}

\subsection*{\textbf{M} - \textit{Login} nel sistema:}

\noindent Come utente voglio poter accedere al sistema e visualizzare la \textit{dashboard} così da gestire i miei progetti precedenti e crearne di nuovi. 

\subsubsection*{Criteri di accettazione:}

\begin{enumerate}
    \item L'utente deve poter accedere inserendo \textit{email} e \textit{password};
    \item In caso di credenziali corrette, l'utente deve essere reindirizzato alla \textit{dashboard} personale;
    \item Se le credenziali sono errate, deve essere mostrato un messaggio di errore appropriato;
    \item Il sistema deve rispettare le politiche di sicurezza, come il blocco temporaneo dopo tentativi falliti ripetuti.
\end{enumerate}

\vspace{0.5cm}

\subsection*{M - Errore di accesso in caso di credenziali errate:}

\noindent Come utente in caso di errore di accesso (es. \textit{password} errata), voglio essere informato con un messaggio chiaro su come risolvere il problema (ad es. come reimpostare la \textit{password}).

\subsubsection*{Criteri di accettazione:}

\begin{enumerate}
    \item Deve essere visualizzato un messaggio chiaro con istruzioni per risolvere il problema (ad esempio, come reimpostare la \textit{password});
    \item Il messaggio di errore non deve rivelare se l'\textit{email} esiste nel sistema per motivi di sicurezza.
\end{enumerate}

\vspace{0.5cm}
\subsection*{M - Piattaforma ottimizzata per \textit{mobile}:}

\noindent Come utente voglio poter accedere alla piattaforma \textit{web} tramite un'interfaccia \textit{mobile} ottimizzata così da poter gestire i miei progetti con facilità anche in mobilità.

\subsubsection*{Criteri di accettazione:}

\begin{enumerate}
    \item La piattaforma deve essere accessibile tramite \textit{browser mobile};
    \item Tutte le funzionalità chiave devono essere usabili su uno schermo \textit{mobile} senza problemi di \textit{layout} o navigazione.
\end{enumerate}

\vspace{0.5cm}

\subsection*{M - \textit{Logout }dal sistema:}

\noindent Come utente voglio avere la possibilità di disconnettermi dal sistema in modo semplice e rapido, così da garantire la sicurezza del mio \textit{account} e prevenire accessi non autorizzati.

\subsubsection*{Criteri di accettazione:}

\begin{enumerate}
    \item L'utente deve potersi disconnettere tramite un pulsante di \textit{logout} ben visibile;
    \item Dopo il \textit{logout}, l'utente deve essere reindirizzato alla pagina di \textit{login};
    \item La sessione dell'utente deve essere invalidata immediatamente.
\end{enumerate}

\vspace{0.5cm}

\section*{\textit{Epic} 2: Gestione dei Progetti}

\subsection*{S - Funzione ricerca e filtraggio progetti:}

\noindent Come utente voglio poter cercare e filtrare i miei progetti per data o nome così da trovare facilmente il progetto desiderato anche in un elenco molto lungo. 

\subsubsection*{Criteri di accettazione:}

\begin{enumerate}
    \item L'utente deve poter cercare i progetti per nome o data tramite un campo di ricerca;
    \item I risultati devono essere aggiornati dinamicamente durante la ricerca;
    \item I filtri devono includere opzioni per ordinare per data di creazione o nome.
\end{enumerate}

\vspace{0.5cm}

\subsection*{M - Visualizzazione singolo progetto}

\noindent Come utente voglio poter selezionare un singolo progetto salvato nella \textit{dashboard} così da visualizzarne i dettagli (nome, stato, data di creazione e le ultime modifiche) e poterlo rigenerare. 

\subsubsection*{Criteri di accettazione:}

\begin{enumerate}
    \item L'utente deve poter cliccare su un progetto dalla lista per accedere alla sua vista dettagliata;
    \item La pagina del progetto deve mostrare nome, stato, data di creazione e tutti i capitoli generati;
    \item Deve essere presente un'opzione per rigenerare il progetto dalla pagina di dettaglio dello stesso.
\end{enumerate}

\vspace{0.5cm}

\subsection*{M - Visualizzazione lista progetto:}

\noindent Come utente voglio poter vedere la lista di tutti i progetti precedentemente generati con annesso nome e data creazione, così da poter scegliere quale selezionare. 

\subsubsection*{Criteri di accettazione:}

\begin{enumerate}
    \item Tutti i progetti dell'utente devono essere visibili in una lista con nome e data di creazione:
    \item La lista deve essere scorrevole e ottimizzata per schermi di diverse dimensioni.
\end{enumerate}

\vspace{0.5cm}

\subsection*{M - Rigenerazione di un progetto:}

\noindent Come utente voglio poter rigenerare un progetto, aggiungendo o modificando informazioni, così da ottenere diverse versioni del documento per meglio adattarsi alle mie esigenze.

\subsubsection*{Criteri di accettazione:}

\begin{enumerate}
    \item L'utente deve poter accedere alla funzione di rigenerazione da un progetto esistente;
    \item Deve essere possibile inserire un \gls{prompt} contenente le direttive da usare per la rigenerazione del progetto;
    \item La versione rigenerata deve essere salvata come nuova versione del progetto.
\end{enumerate}

\vspace{0.5cm}

\subsection*{C - Visualizzazione versioni precedenti progetto:}

\noindent Come utente voglio poter visualizzare le versioni precedenti di un progetto così da poter ripristinare o confrontare modifiche, migliorando il controllo sulle evoluzioni del documento.

\subsubsection*{Criteri di accettazione:}

\begin{enumerate}
    \item L'utente deve poter visualizzare una lista cronologica delle versioni precedenti del progetto;
    \item Deve essere possibile selezionare e visualizzare una versione precedente.
\end{enumerate}

\vspace{0.5cm}
\pagebreak
\subsection*{S - Rigenerazione specifici capitoli}

\noindent Come utente voglio rigenerare specifici capitoli di un progetto già creato, mantenendo inalterati gli altri capitoli, così da aggiornare solo le sezioni che richiedono modifiche senza dover rigenerare tutto il documento.

\subsubsection*{Criteri di accettazione:}

\begin{enumerate}
    \item L'utente deve poter selezionare il capitolo da rigenerare dalla pagina di visualizzazione singolo progetto;
    \item Deve essere possibile inserire un \gls{prompt} contenente le direttive da usare per la rigenerazione del capitolo;
    \item La versione rigenerata deve essere salvata come nuova versione del progetto;
    \item I capitoli non selezionati devono rimanere invariati.
\end{enumerate}

\vspace{0.5cm}

\subsection*{C - Rimozione capitoli non necessari:}

\noindent Come utente voglio poter rimuovere singoli capitoli così da eliminare contenuti non necessari dal progetto. 

\subsubsection*{Criteri di accettazione:}

\begin{enumerate}
    \item L'utente deve poter selezionare e rimuovere i capitoli dal progetto;
    \item Prima della rimozione definitiva, deve essere mostrata una finestra di conferma.
\end{enumerate}

\vspace{0.5cm}

\subsection*{M - Eliminazione di un progetto:}

\noindent Come utente, voglio poter eliminare un progetto che ho creato in modo che possa rimuovere progetti non più necessari o irrilevanti.

\subsubsection*{Criteri di accettazione:}

\begin{enumerate}
    \item L'utente dalla pagina di visualizzazione singolo progetto deve poter richiedere l’eliminazione dello stesso;
    \item Deve essere mostrata una finestra di conferma prima dell'eliminazione definitiva del progetto;
    \item L'eliminazione deve rimuovere tutti i dati associati al progetto senza lasciare tracce;
    \item Se l'eliminazione è completata con successo, l'utente deve ricevere una notifica di conferma.
\end{enumerate}

\vspace{0.5cm}

\section*{\textit{Epic} 3: Creazione di nuovi progetti}

\subsection*{M - Generazione progetto tramite \textit{preset}:}

\noindent Come utente voglio poter generare un progetto con l'aiuto di \textit{preset} predefiniti e ottimizzati, così da creare rapidamente un documento utile e strutturato senza partire da zero.

\subsubsection*{Criteri di accettazione:}

\begin{enumerate}
    \item All’interno della pagina di creazione di un progetto l'utente deve poter selezionare un \textit{preset} predefinito dalla lista;
    \item Dopo la selezione e la compilazione del \textit{preset} il progetto deve essere generato con i dati inseriti;
    \item Se la generazione avviene con successo, l’utente viene reindirizzato alla pagina di visualizzazione del singolo progetto;
    \item L’utente riceve una notifica di successo.
\end{enumerate}

\vspace{0.5cm}

\subsection*{M - Visualizzazione descrizione \textit{preset}:}

\noindent Come utente voglio visualizzare una descrizione dettagliata di ciascun \textit{preset} così da scegliere quello più adatto alle mie esigenze specifiche.

\subsubsection*{Criteri di accettazione:}

\begin{enumerate}
    \item Ogni \textit{preset} deve avere una descrizione dettagliata visibile prima della selezione;
    \item La descrizione deve includere le caratteristiche e i casi d'uso principali del \textit{preset}.
\end{enumerate}

\vspace{0.5cm}

\subsection*{S - Salvataggio \textit{preset} compilato:}

\noindent Come utente voglio poter salvare il \textit{preset} compilato così da poterlo completare e generare il documento in un momento successivo.

\subsubsection*{Criteri di accettazione:}

\begin{enumerate}
    \item L'utente deve poter salvare un \textit{preset} (parzialmente) compilato per completarlo in un secondo momento;
    \item Il sistema deve indicare chiaramente che il \textit{preset} è salvato in bozza.
\end{enumerate}

\vspace{0.5cm}

\pagebreak
\subsection*{C - Personalizzazione campi \textit{preset}:}

\noindent Come utente voglio personalizzare i campi del \textit{preset} così da adattarlo alle mie specifiche necessità e rendere unico il progetto. 

\subsubsection*{Criteri di accettazione:}

\begin{enumerate}
    \item L'utente deve poter modificare i campi predefiniti del \textit{preset};
    \item Le modifiche devono essere salvate, rendendo possibile l’utilizzo del \textit{preset} personalizzato in secondo momento
\end{enumerate}

\vspace{0.5cm}

\subsection*{C - Generazione progetto senza \textit{preset}:}

\noindent Come utente voglio poter generare un progetto senza l’utilizzo di un \textit{preset} così da avere il massimo controllo e flessibilità sulla struttura del documento finale, migliorando la personalizzazione per casi d'uso complessi. 

\subsubsection*{Criteri di accettazione:}

\begin{enumerate}
    \item L'utente deve poter generare un nuovo progetto senza scegliere un \textit{preset};
    \item Deve essere possibile inserire un prompt contenente le direttive da usare per la generazione del progetto.
\end{enumerate}

\vspace{0.5cm}

\subsection*{M - Download documento progetto:}

\noindent Come utente voglio poter scaricare il progetto generato in formato PDF così da poterlo condividere facilmente con il mio \textit{team} o archiviare per un utilizzo futuro.

\subsubsection*{Criteri di accettazione:}

\begin{enumerate}
    \item Dalla pagina di visualizzazione singolo progetto l'utente deve poter scaricare il progetto in formato PDF;
    \item Il \textit{file} generato deve includere tutti i capitoli del progetto.
\end{enumerate}

\vspace{0.5cm}

\subsection*{S - Generazione progetto tramite diversi \gls{llm}:}

\noindent Come utente voglio poter scegliere tra vari \gls{llm} per la generazione del progetto così da avere varie maggiore scelta (progetti più grandi vs. progetti più piccoli). 

\subsubsection*{Criteri di accettazione:}

\begin{enumerate}
    \item L'utente deve poter selezionare tra una lista di \gls{llm} disponibili prima della generazione del progetto;
    \item Ogni modello deve avere una descrizione che ne spiega le capacità ed i casi d'uso.
\end{enumerate}

\vspace{0.5cm}

\section*{\textit{Epic} 4: Piattaforma amministratore}

\subsection*{C - Aggiunta nuovo \textit{preset}:}

\noindent Come amministratore voglio poter aggiungere un nuovo \textit{preset} così da poter dare maggior scelta nella creazione di progetti agli utenti.

\subsubsection*{Criteri di accettazione:}

\begin{enumerate}
    \item L'amministratore deve poter accedere a un modulo per creare un nuovo \textit{preset};
    \item Il \textit{preset} aggiunto deve essere visibile agli utenti finali immediatamente dopo il salvataggio.
\end{enumerate}

\vspace{0.5cm}

\subsection*{C - Modifica \textit{preset} esistente:}

\noindent Come amministratore voglio poter modificare un \textit{preset} già esistente, così da poterlo migliorare e aggiornare nel tempo o non renderlo più disponibile all’utilizzo da parte degli utenti.

\subsubsection*{Criteri di accettazione:}

\begin{enumerate}
    \item L'amministratore deve poter modificare nome, descrizione e campi di un \textit{preset} esistente;
    \item Le modifiche devono essere salvate e applicate ai nuovi progetti creati con quel \textit{preset}.
\end{enumerate}

\vspace{0.5cm}

\subsection*{C - Eliminazione \textit{preset} esistente:}

\noindent Come amministratore voglio poter eliminare un \textit{preset} già esistente così da poter eliminare \textit{preset} poco utilizzati o obsoleti.

\begin{enumerate}
    \item L'amministratore deve poter eliminare un \textit{preset} selezionato;
    \item Deve essere mostrata una finestra di conferma prima della rimozione definitiva.
\end{enumerate}

\vspace{0.5cm}

\subsection*{C - Gestione \gls{llm}:}

\noindent Come amministratore voglio poter aggiungere o rimuovere l’accesso agli utenti ad una \gls{llm}, così da poter eliminare \gls{llm} obsolete o aggiungerne di nuove maggiormente performanti.

\subsubsection*{Criteri di accettazione:}

\begin{enumerate}
    \item L'amministratore deve poter aggiungere nuovi modelli \gls{llm} alla piattaforma;
    \item Deve essere possibile rimuovere i modelli \gls{llm} non più utilizzati.
\end{enumerate}