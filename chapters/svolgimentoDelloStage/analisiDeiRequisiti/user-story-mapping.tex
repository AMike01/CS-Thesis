\subsection*{M - Generazione progetto tramite \textit{preset}:}

\noindent Come utente voglio poter generare un progetto con l'aiuto di \textit{preset} predefiniti e ottimizzati, così da creare rapidamente un documento utile e strutturato senza partire da zero.

\subsubsection*{Criteri di accettazione:}

\begin{enumerate}
    \item All’interno della pagina di creazione di un progetto l'utente deve poter selezionare un \textit{preset} predefinito dalla lista;
    \item Dopo la selezione e la compilazione del \textit{preset} il progetto deve essere generato con i dati inseriti;
    \item Se la generazione avviene con successo, l’utente viene reindirizzato alla pagina di visualizzazione del singolo progetto;
    \item L’utente riceve una notifica di successo.
\end{enumerate}

\vspace{0.2cm}

\subsection*{M - Download documento progetto:}

\noindent Come utente voglio poter scaricare il progetto generato in formato PDF così da poterlo condividere facilmente con il mio \textit{team} o archiviare per un utilizzo futuro.

\subsubsection*{Criteri di accettazione:}

\begin{enumerate}
    \item Dalla pagina di visualizzazione singolo progetto l'utente deve poter scaricare il progetto in formato PDF;
    \item Il \textit{file} generato deve includere tutti i capitoli del progetto.
\end{enumerate}

\vspace{0.2cm}

\subsection*{M - Rigenerazione di un progetto:}

\noindent Come utente voglio poter rigenerare un progetto, aggiungendo o modificando informazioni, così da ottenere diverse versioni del documento per meglio adattarsi alle mie esigenze.

\subsubsection*{Criteri di accettazione:}

\begin{enumerate}
    \item L'utente deve poter accedere alla funzione di rigenerazione da un progetto esistente;
    \item Deve essere possibile inserire un \gls{prompt} contenente le direttive da usare per la rigenerazione del progetto;
    \item La versione rigenerata deve essere salvata come nuova versione del progetto.
\end{enumerate}



