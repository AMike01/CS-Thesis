\section{Analisi dei requisiti}
\label{sez:analisi-dei-requisiti}

Spiegazione di cos'è l'analisi dei requisiti e di come è stata effettuata.\\

\subsection{\textit{User Story Mapping}}
\label{subsec:user-story-mapping}

\subsubsection{Introduzione}
\label{subsubsec:introduzione}

Introduzione allo User Story Mapping e spiegazione di come è stato utilizzato, per cosa risulta utile, ecc.\\

\subsubsection{\textit{Epic} e \textit{stories} identificate}
\label{subsubsec:epic-stories}

Descrizione degli Epic e delle stories identificate durante l'analisi dei requisiti.\\

\subsection{Casi d'uso}
\label{subsec:casi-duso}

\subsubsection{Attori}
\label{subsubsec:attori}

Descrizione degli attori individuati durante l'analisi dei requisiti.\\

\subsubsection{Casi d'uso individuati}
\label{subsubsec:casi-uso-individuati}

Descrizione dei casi d'uso individuati durante l'analisi dei requisiti.\\

\subsection{Tracciamento dei requisiti}
\label{subsec:requisiti}

Lista dei requisiti individuati, con spiegazione della notazione utilizzata e classificazione tra requisiti funzionali e non funzionali, obbligatori, desiderabili e facoltativi.\\