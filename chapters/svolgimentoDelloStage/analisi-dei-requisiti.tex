\pagebreak
\section{Analisi dei requisiti}
\label{sez:analisi-dei-requisiti}
L’analisi dei requisiti rappresenta una fase cruciale in qualsiasi progetto \textit{software}, poiché consente di definire in modo dettagliato e strutturato ciò che il sistema dovrà realizzare per soddisfare le necessità degli utenti e degli \textit{stakeholder}.\\
In questo progetto, l’analisi è stata condotta seguendo un approccio metodico che si è articolato in diverse fasi.\\

\noindent In primo luogo, ho effettuato una mappatura delle \gls{user-stories}, un processo che ha permesso di identificare le funzionalità principali richieste dagli utenti attraverso una rappresentazione visuale e iterativa. \\
Successivamente, ho individuato i casi d’uso, descrivendo in dettaglio gli scenari di interazione tra gli attori del sistema e le funzionalità del \textit{software}.\\
Infine, sulla base di queste informazioni, ho creato i \textit{requisiti}, formalizzando in modo chiaro e specifico ciò che il sistema deve implementare, suddividendo i requisiti in funzionali e non funzionali per garantire una visione completa del progetto.
\subsection{\textit{User Story Mapping}}
\label{subsec:user-story-mapping}

\subsection*{Introduzione}
\label{subsubsec:introduzione}

Le \gls{user-stories} rappresentano una tecnica fondamentale nel contesto dello sviluppo \textit{Agile}, in particolare all'interno del \textit{framework} \textit{Scrum}. \\
Questa metodologia consente di creare una visione d'insieme delle funzionalità richieste da un sistema, organizzandole in modo strutturato e collaborativo\footcite{site:user-stories}. \\

\noindent Lo \textit{User Story Mapping} è uno strumento visuale che suddivide il lavoro in componenti incrementali, aiutando il \textit{team} a comprendere il percorso del cliente e a pianificare il rilascio di valore progressivo.\\

\subsection*{Utilizzo nel contesto \textit{Scrum}}\\

\noindent Nel contesto di \textit{Scrum}, le \gls{user-stories} sono utilizzate per:

\begin{itemize}
    \item \textbf{Comprendere le esigenze degli utenti:} La mappatura aiuta a identificare le principali funzionalità del prodotto dal punto di vista degli utenti finali;
    \item \textbf{Definire un \textit{backlog} strutturato:} La tecnica consente di creare un \gls{product-backlog} organizzato, ordinando le \gls{user-stories} per priorità e obiettivi di rilascio;
    \item \textbf{Pianificare gli \gls{sprint}:} La mappa aiuta il \textit{Product Owner} ed il \textit{team} ad individuare le storie da completare in ciascun \gls{sprint}, bilanciando valore e complessità;
    \item \textbf{Favorire la collaborazione:} La mappa è creata in un incontro collaborativo che coinvolge il \textit{Product Owner}, il \textit{team} di sviluppo e, quando possibile, gli \textit{stakeholder}, per garantire un allineamento sulle priorità.
\end{itemize}

\pagebreak
\subsection*{Struttura delle \gls{user-stories}}\\

\noindent Ogni \textit{user story} segue una struttura standard che include diversi elementi essenziali:\\
\begin{itemize}

    \item \textbf{Priorità}, ogni storia viene classificata in base alla sua priorità, indicata con tre livelli derivati dal metodo \textit{MoSCoW}\footcite{site:moscow-method}:
    \begin{itemize}
        \item \textbf{M:} \textit{Must have} (essenziale): funzionalità indispensabili per il sistema;
        \item \textbf{S:} \textit{Should have} (importante): funzionalità utili ma non critiche;
        \item \textbf{C:} \textit{Could have} (opzionale): funzionalità che aggiungono valore ma possono essere escluse in caso di limitazioni temporali o di risorse.   
    \end{itemize}

    \item \textbf{Descrizione}, ogni storia è descritta in modo sintetico attraverso la notazione standard:

    \begin{center}
        Come [\textit{utente}], voglio [\textit{azione}], affinché [\textit{obiettivo}].
    \end{center}
    Ad esempio: "Come cliente, voglio visualizzare il catalogo prodotti, affinché possa scegliere cosa acquistare."\\

    \item \textbf{Criteri di accettazione}, questi definiscono le condizioni che devono essere soddisfatte affinché le \gls{user-stories} possano essere considerate completate.\\
    Sono specifiche misurabili che servono come riferimento per i \textit{test} e l’approvazione. \\

    \noindent Ad esempio, per la storia sopra, i criteri di accettazione potrebbero essere:

    \begin{itemize}
        \item Il catalogo deve mostrare almeno 20 prodotti per pagina;
        \item Ogni prodotto deve avere nome, immagine e prezzo visibili;
        \item L’utente deve poter filtrare i prodotti per categoria.
    \end{itemize}

\end{itemize}

\noindent Questa combinazione di priorità, descrizione e criteri di accettazione rende le \gls{user-stories} uno strumento potente per pianificare, monitorare e validare lo sviluppo di un progetto, assicurando che tutte le funzionalità siano orientate al valore per l’utente e il \textit{business}.

\subsection*{\textit{Epic} e \textit{stories} identificate}
\label{subsubsec:epic-stories}

Descrizione degli Epic e delle stories identificate durante l'analisi dei requisiti.\\

\subsection{Casi d'uso}
\label{subsec:casi-duso}

\subsubsection{Attori}
\label{subsubsec:attori}

Descrizione degli attori individuati durante l'analisi dei requisiti.\\

\subsubsection{Casi d'uso individuati}
\label{subsubsec:casi-uso-individuati}

Descrizione dei casi d'uso individuati durante l'analisi dei requisiti.\\

\subsection{Tracciamento dei requisiti}
\label{subsec:requisiti}

Lista dei requisiti individuati, con spiegazione della notazione utilizzata e classificazione tra requisiti funzionali e non funzionali, obbligatori, desiderabili e facoltativi.\\