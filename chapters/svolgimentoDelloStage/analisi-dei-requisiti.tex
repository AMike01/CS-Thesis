\pagebreak
\section{Analisi dei requisiti}
\label{sez:analisi-dei-requisiti}
L’analisi dei requisiti rappresenta un'attività cruciale in qualsiasi progetto \textit{software}, poiché consente di definire in modo dettagliato e strutturato ciò che il sistema dovrà realizzare per soddisfare le necessità degli utenti e degli \textit{stakeholder}.\\
In questo progetto, ho condotto l’analisi seguendo un approccio metodico che si è articolato in diverse fasi.\\

\noindent In primo luogo, ho effettuato una mappatura delle \gls{user-stories}, un processo che ha permesso di identificare le funzionalità principali richieste dagli utenti attraverso una rappresentazione visuale e iterativa. \\
Successivamente, ho individuato i casi d’uso, descrivendo in dettaglio gli scenari di interazione tra gli attori del sistema e le funzionalità del \textit{software}.\\
Infine, sulla base di queste informazioni, ho creato i \textit{requisiti}, formalizzando in modo chiaro e specifico ciò che il sistema deve implementare, suddividendo i requisiti in funzionali, non funzionali e tecnologici per garantire una visione completa del progetto.
\subsection{\textit{User story mapping}}
\label{subsec:user-story-mapping}

\subsection*{Introduzione}
\label{subsubsec:introduzione}

Le \gls{user-stories} rappresentano una tecnica fondamentale nel contesto dello sviluppo \textit{Agile}, in particolare all'interno del \textit{framework} \textit{Scrum}. \\
Questa metodologia mi ha consentito di creare una visione d'insieme delle funzionalità richieste dal sistema, organizzandole in modo strutturato e collaborativo\footcite{site:user-stories}. \\

\subsection*{Utilizzo nel contesto \textit{scrum}}\\

\noindent Nel contesto di \textit{Scrum}, le \gls{user-stories} sono utilizzate per:

\begin{itemize}
    \item \textbf{Comprendere le esigenze degli utenti:} La mappatura aiuta a identificare le principali funzionalità del prodotto dal punto di vista degli utenti finali;
    \item \textbf{Definire un \textit{backlog} strutturato:} La tecnica consente di creare un \gls{product-backlog} organizzato, ordinando le \gls{user-stories} per priorità e obiettivi di rilascio;
    \item \textbf{Pianificare gli \gls{sprint}:} La mappa aiuta il \textit{Product Owner} ed il \textit{team} ad individuare le storie da completare in ciascun \gls{sprint}, bilanciando valore e complessità;
    \item \textbf{Favorire la collaborazione:} La mappa è creata in un incontro collaborativo che coinvolge il \textit{Product Owner}, il \textit{team} di sviluppo e, quando possibile, gli \textit{stakeholder}, per garantire un allineamento sulle priorità.
\end{itemize}

\pagebreak
\subsection*{Struttura delle \textit{user stories}}\\

\noindent Ogni \textit{user story} segue una struttura standard che include diversi elementi essenziali:\\
\begin{itemize}

    \item \textbf{Priorità}, ho classificato ogni storia in base alla sua priorità, indicata con tre livelli derivati dal metodo \textit{MoSCoW}\footcite{site:moscow-method}:
    \begin{itemize}
        \item \textbf{M:} \textit{Must have} (essenziale): funzionalità indispensabili per il sistema;
        \item \textbf{S:} \textit{Should have} (importante): funzionalità utili ma non critiche;
        \item \textbf{C:} \textit{Could have} (opzionale): funzionalità che aggiungono valore ma possono essere escluse in caso di limitazioni temporali o di risorse.   
    \end{itemize}

    \item \textbf{Descrizione}, ho descritto ogni storia in modo sintetico attraverso la notazione standard:

    \begin{center}
        Come [\textit{utente}], voglio [\textit{azione}], affinché [\textit{obiettivo}].
    \end{center}
    Ad esempio: "Come cliente, voglio visualizzare il catalogo prodotti, affinché possa scegliere cosa acquistare."\\

    \item \textbf{Criteri di accettazione}, ho definito le condizioni che devono essere soddisfatte affinché le \gls{user-stories} possano essere considerate completate.\\
    Sono specifiche misurabili che servono come riferimento per i \textit{test} e l’approvazione. \\

    \noindent Ad esempio, per la storia sopra, i criteri di accettazione potrebbero essere:

    \begin{itemize}
        \item Il catalogo deve mostrare almeno 20 prodotti per pagina;
        \item Ogni prodotto deve avere nome, immagine e prezzo visibili;
        \item L’utente deve poter filtrare i prodotti per categoria.
    \end{itemize}

\end{itemize}

\pagebreak
\subsection*{\textit{User stories} identificate}

Di seguito le \gls{user-stories} più significative che ho individuato durante l'analisi dei requisiti:
\label{subsubsec:epic-stories}

\subsection*{M - Generazione progetto tramite \textit{preset}:}

\noindent Come utente voglio poter generare un progetto con l'aiuto di \textit{preset} predefiniti e ottimizzati, così da creare rapidamente un documento utile e strutturato senza partire da zero.

\subsubsection*{Criteri di accettazione:}

\begin{enumerate}
    \item All’interno della pagina di creazione di un progetto l'utente deve poter selezionare un \textit{preset} predefinito dalla lista;
    \item Dopo la selezione e la compilazione del \textit{preset} il progetto deve essere generato con i dati inseriti;
    \item Se la generazione avviene con successo, l’utente viene reindirizzato alla pagina di visualizzazione del singolo progetto;
    \item L’utente riceve una notifica di successo.
\end{enumerate}

\vspace{0.2cm}

\subsection*{M - Download documento progetto:}

\noindent Come utente voglio poter scaricare il progetto generato in formato PDF così da poterlo condividere facilmente con il mio \textit{team} o archiviare per un utilizzo futuro.

\subsubsection*{Criteri di accettazione:}

\begin{enumerate}
    \item Dalla pagina di visualizzazione singolo progetto l'utente deve poter scaricare il progetto in formato PDF;
    \item Il \textit{file} generato deve includere tutti i capitoli del progetto.
\end{enumerate}

\vspace{0.2cm}

\subsection*{M - Rigenerazione di un progetto:}

\noindent Come utente voglio poter rigenerare un progetto, aggiungendo o modificando informazioni, così da ottenere diverse versioni del documento per meglio adattarsi alle mie esigenze.

\subsubsection*{Criteri di accettazione:}

\begin{enumerate}
    \item L'utente deve poter accedere alla funzione di rigenerazione da un progetto esistente;
    \item Deve essere possibile inserire un \gls{prompt} contenente le direttive da usare per la rigenerazione del progetto;
    \item La versione rigenerata deve essere salvata come nuova versione del progetto.
\end{enumerate}





\subsection{Casi d'uso}
\label{subsec:casi-duso}

I diagrammi dei casi d’uso consentono di descrivere in modo chiaro e strutturato le interazioni tra gli attori e il sistema.\\

\noindent Per la stesura dei casi d’uso, ho adottato una convenzione specifica, organizzata come segue:

\begin{center}
\textbf{UC[Numero] - Nome del caso d’uso}
\end{center}

Dove:
\begin{itemize}
    \item \textbf{UC}: acronimo di \textit{Use Case};
    \item \textbf{Numero}: identificatore progressivo del caso d’uso.\\
    I sottocasi sono rappresentati nella forma [Numero].[SottoNumero].
\end{itemize}

\noindent Ogni caso d’uso è corredato da una descrizione dettagliata, dalla lista degli attori coinvolti, dalla sua descrizione, dalle precondizioni e dalle postcondizioni necessarie affinché il caso d’uso possa essere eseguito correttamente.\\

\noindent Ho incluso esclusivamente i casi d’uso principali, omettendo quelli banali o secondari per privilegiare chiarezza e rilevanza.\\

\subsubsection{Attori}
\label{subsubsec:attori}

Nel contesto dei casi d'uso, un attore è colui che interagisce con il sistema per ottenere un obiettivo specifico.\\
Uno schema contenente tutti gli attori è visibile in {\hyperref[fig:attori-casi-duso]{Figura 3.2}}, mentre i dettagli di ciascun attore sono descritti di seguito:

\subsection*{Utente non autenticato:}

\begin{itemize}
    \item \textbf{Descrizione:}  L'utente non autenticato è una persona che sta cercando di accedere al sistema, ma non ha effettuato il \textit{login}.\\
    Non ha accesso ai dati sensibili o alle funzionalità avanzate riservate agli utenti autenticati;
    \item \textbf{Ruolo:} Questo attore ha il compito di interagire con il sistema senza diritti di accesso completi.\\
    Il suo obiettivo principale è autenticarsi per diventare un utente con pieno accesso al sistema;
    \item \textbf{Attività principali:}
        \begin{itemize}
            \item Effettuare il \textit{login} con le proprie credenziali;
            \item Visualizzare i messaggi di errore in caso di \textit{login} fallito.
        \end{itemize}
\end{itemize}

\subsection*{Utente autenticato:}

\begin{itemize}
    \item \textbf{Descrizione:}  L'utente autenticato è una persona che ha effettuato il \textit{login} con successo ed ha accesso completo alle funzionalità del sistema.\\
    Può cercare, filtrare, visualizzare, generare e rigenerare i progetti;
    \item \textbf{Ruolo:} Questo attore può eseguire operazioni avanzate, come la generazione di progetti, il salvataggio delle bozze ed il \textit{download} dei progetti in formato PDF;
    \item \textbf{Attività principali:}
        \begin{itemize}
            \item Visualizzare e cercare progetti;
            \item Generare e rigenerare progetti;
            \item Salvare e visualizzare bozze;
            \item Visualizzare e scaricare progetti in formato PDF.
        \end{itemize}
\end{itemize}

\subsection*{\gls{llmg}:}

\begin{itemize}
    \item \textbf{Descrizione:}  L'attore \gls{llm} è un sistema automatizzato di generazione dei contenuti, che utilizza algoritmi avanzati di \gls{aig} per generare e rigenerare progetti o singoli capitoli di progetti. \\
    Il \gls{llm} è responsabile dell'elaborazione delle richieste degli utenti e della generazione automatica del contenuto tecnico dei progetti;
    \item \textbf{Ruolo:}  Il \gls{llm} è un componente del sistema che esegue la generazione di contenuti complessi in modo autonomo.\\
    La sua funzione principale è rispondere alle richieste dell'utente riguardo la creazione e la rigenerazione di progetti;
    \item \textbf{Attività principali:}
        \begin{itemize}
            \item Ricevere richieste di generazione di progetti da parte degli utenti;
            \item Generare contenuti basati su preset predefiniti;
            \item Rigenerare progetti o singoli capitoli in risposta a richieste di aggiornamento;
            \item Restituire i contenuti generati all'utente.
        \end{itemize}
\end{itemize}

\begin{figure}[H]
    \centering
    \includegraphics[width=1\textwidth]{usecase/attori.png}
    \caption{Attori dei casi d'uso}
    \label{fig:attori-casi-duso}
\end{figure}




\pagebreak
\subsubsection{Casi d'uso individuati}
\label{subsubsec:casi-uso-individuati}

\section*{UC1 - \textit{Login} dell'utente}

\begin{figure}[H]
    \centering
    \includegraphics[scale=0.5]{usecase/uc1.png}
    \caption{UC1 - \textit{Login} dell'utente}
    \label{fig:uc1}
\end{figure}
\begin{itemize}
    \item \textbf{Attori}: Utente non autenticato;
    \item \textbf{Descrizione}: L'utente inserisce le proprie credenziali per accedere al sistema;
    \item \textbf{Precondizioni}: L'utente non è autenticato nel sistema;
    \item \textbf{Postcondizioni}: L'utente viene autenticato e reindirizzato alla \textit{dashboard} del sistema;
    \item \textbf{Flusso principale}:
    \begin{enumerate}
        \item L'utente naviga alla pagina di \textit{login};
        \item L'utente inserisce \textit{email} e \textit{password};
        \item Il sistema verifica le credenziali;
        \item Se le credenziali sono corrette, l'utente viene autenticato e reindirizzato alla \textit{dashboard}.
    \end{enumerate}
    \item \textbf{Estensioni}: UC3 - Visualizzazione errore in caso di \textit{login} non riuscito
\end{itemize}

\vspace{0.5cm}
\section*{UC2 - \textit{Logout} dell'utente}
\begin{itemize}
    \item \textbf{Attori}: Utente autenticato;
    \item \textbf{Descrizione}: L'utente può disconnettersi dal sistema;
    \item \textbf{Precondizioni}: L'utente è autenticato;
    \item \textbf{Postcondizioni}: L'utente viene disconnesso e reindirizzato alla pagina di \textit{login};
    \item \textbf{Flusso principale}:
    \begin{enumerate}
        \item L'utente clicca sul pulsante di \textit{logout};
        \item L'utente conferma la volontà di effettuare il \textit{logout};
        \item Il sistema disconnette l'utente;
        \item L'utente viene reindirizzato alla pagina di \textit{login}.
    \end{enumerate}
\end{itemize}

\vspace{0.5cm}
\section*{UC3 - Visualizzazione errore in caso di \textit{login} non riuscito}
\begin{itemize}
    \item \textbf{Attori}: Utente non autenticato;
    \item \textbf{Descrizione}: L'utente visualizza un messaggio di errore quando il \textit{login} fallisce;
    \item \textbf{Precondizioni}: L'utente ha inserito credenziali errate;
    \item \textbf{Postcondizioni}: Un messaggio di errore viene mostrato all'utente;
    \item \textbf{Flusso principale}:
    \begin{enumerate}
        \item L'utente inserisce credenziali errate;
        \item Il sistema verifica le credenziali;
        \item Il sistema visualizza un messaggio di errore indicando che il \textit{login} è fallito.
    \end{enumerate}
\end{itemize}

\vspace{0.5cm}  
\section*{UC4 - Visualizzazione lista progetti}
\begin{itemize}
    \item \textbf{Attori}: Utente autenticato;
    \item \textbf{Descrizione}: L'utente visualizza la lista di tutti i progetti disponibili;
    \item \textbf{Precondizioni}: L'utente è autenticato;
    \item \textbf{Postcondizioni}: La lista di progetti viene visualizzata;
    \item \textbf{Flusso principale}:
    \begin{enumerate}
        \item L'utente accede alla \textit{dashboard} del sistema;
        \item Il sistema recupera e visualizza la lista di tutti i progetti disponibili;
        \item L'utente può vedere il nome, la data di creazione ed il preset usato per la generazione dei progetti.
    \end{enumerate}
\end{itemize}

\vspace{0.5cm}  
\section*{UC5 - Ricerca e filtraggio dei progetti}
\begin{itemize}
    \item \textbf{Attori}: Utente autenticato;
    \item \textbf{Descrizione}: L'utente cerca e filtra i progetti utilizzando specifici criteri;
    \item \textbf{Precondizioni}: L'utente è autenticato e si trova nella dashboard di sistema;
    \item \textbf{Postcondizioni}: I progetti che soddisfano i criteri di ricerca e/o filtro vengono visualizzati;
    \item \textbf{Flusso principale}:
    \begin{enumerate}
        \item L'utente inserisce parole chiave e/o seleziona i filtri;
        \item Il sistema applica i filtri e visualizza i progetti corrispondenti.
    \end{enumerate}
\end{itemize}

\vspace{0.5cm}  
\section*{UC6 - Visualizzazione dei dettagli di un progetto singolo}
\begin{itemize}
    \item \textbf{Attori}: Utente autenticato;
    \item \textbf{Descrizione}: L'utente visualizza i dettagli di un progetto specifico selezionato dalla lista dei progetti;
    \item \textbf{Precondizioni}: 
    \begin{itemize}
        \item L'utente è autenticato;
        \item Il progetto selezionato è stato precedentemente creato dall'utente.
    \end{itemize}
    \item \textbf{Postcondizioni}: Il sistema mostra i dettagli del progetto selezionato, inclusi nome, descrizione, data di creazione, e tutti i capitoli che lo compongono;
    \item \textbf{Flusso principale}:
    \begin{enumerate}
        \item L'utente accede alla lista dei progetti salvati;
        \item L'utente seleziona un progetto dalla lista;
        \item Il sistema recupera i dati del progetto selezionato dal \textit{database};
        \item Il sistema mostra i dettagli del progetto in una pagina dedicata, inclusi:
        \begin{itemize}
            \item Nome del progetto;
            \item Descrizione del progetto;
            \item Data di creazione;
            \item Sezioni e contenuti specifici del progetto.
        \end{itemize}
    \end{enumerate}
\end{itemize}

\vspace{0.5cm}  
\section*{UC7 - Visualizzazione descrizione \textit{preset}}
\begin{itemize}
    \item \textbf{Attori}: Utente autenticato;
    \item \textbf{Descrizione}: L'utente visualizza la descrizione di un \textit{preset} disponibile;
    \item \textbf{Precondizioni}: L'utente è autenticato;
    \item \textbf{Postcondizioni}: La descrizione del \textit{preset} viene visualizzata;
    \item \textbf{Flusso principale}:
    \begin{enumerate}
        \item L’utente naviga nella pagina di creazione di un progetto;
        \item L'utente seleziona un \textit{preset} dalla lista di quelli disponibili;
        \item Il sistema visualizza la descrizione dettagliata del \textit{preset}.
    \end{enumerate}
\end{itemize}

\vspace{0.5cm}  
\section*{UC8 - Salvataggio di un \textit{preset} compilato}
\begin{itemize}
    \item \textbf{Attori}: Utente autenticato;
    \item \textbf{Descrizione}: L'utente salva un \textit{preset} parzialmente compilato come bozza;
    \item \textbf{Precondizioni}: L'utente è autenticato e deve aver iniziato a compilare un \textit{preset};
    \item \textbf{Postcondizioni}: Il \textit{preset} parzialmente compilato viene salvato come bozza;
    \item \textbf{Flusso principale}:
    \begin{enumerate}
        \item L'utente compila una parte del \textit{preset};
        \item L'utente clicca su "Salva bozza";
        \item Il sistema salva la bozza del \textit{preset} nel \textit{database}.
    \end{enumerate}
\end{itemize}

\vspace{0.5cm}  
\section*{UC9 - Visualizzazione lista bozze}
\begin{itemize}
    \item \textbf{Attori}: Utente autenticato;
    \item \textbf{Descrizione}: L'utente visualizza la lista di tutte le bozze disponibili;
    \item \textbf{Precondizioni}: L'utente è autenticato;
    \item \textbf{Postcondizioni}: La lista di bozze viene visualizzata;
    \item \textbf{Flusso principale}:
    \begin{enumerate}
        \item L'utente accede alla \textit{dashboard} del sistema;
        \item L'utente seleziona il pulsante di visualizzazione della lista di bozze;
        \item Il sistema recupera e visualizza la lista di tutte le bozze disponibili;
        \item L'utente può vedere il nome, la data di creazione ed il preset utilizzato per ogni bozza.
    \end{enumerate}
\end{itemize}

\vspace{0.5cm}  
\section*{UC10 - Ricerca e filtraggio delle bozze}
\begin{itemize}
    \item \textbf{Attori}: Utente autenticato;
    \item \textbf{Descrizione}: L'utente cerca e filtra le bozze utilizzando specifici criteri;
    \item \textbf{Precondizioni}: 
    \begin{itemize}
        \item L'utente è autenticato;
        \item L'utente si trova nella \textit{dashboard} di sistema ed ha selezionato il pulsante di visualizzazione della lista di bozze.
    \end{itemize}
    \item \textbf{Postcondizioni}: Le bozze che soddisfano i criteri di ricerca e/o filtro vengono visualizzate;
    \item \textbf{Flusso principale}:
    \begin{enumerate}
        \item L'utente inserisce parole chiave e/o seleziona i filtri;
        \item Il sistema applica i filtri e visualizza le bozze corrispondenti.
    \end{enumerate}
\end{itemize}

\vspace{0.5cm}  
\section*{UC11 - Generazione di un progetto}
\begin{figure}[H]
    \centering
    \includegraphics[scale=0.6]{usecase/uc11.png}
    \caption{UC11 - Generazione di un progetto}
    \label{fig:uc11}
\end{figure}
\begin{itemize}
    \item \textbf{Attori}: Utente autenticato, \gls{llm};
    \item \textbf{Descrizione}: L'utente avvia la generazione di un progetto utilizzando un \textit{preset};
    \item \textbf{Precondizioni}: 
    \begin{itemize}
        \item L'utente è autenticato;
        \item L'utente si trova nella pagina di creazione di un progetto ed ha selezionato un \textit{preset}.
    \end{itemize}
    \item \textbf{Postcondizioni}: Il progetto viene generato e l'utente vede le informazioni generate;
    \item \textbf{Flusso principale}:
    \begin{enumerate}
        \item L'utente compila nella sua interezza il \textit{preset} selezionato;
        \item L'utente richiede la generazione del progetto;
        \item Il sistema invia la richiesta al \gls{llm};
        \item Il \gls{llm} elabora la richiesta e genera il progetto;
        \item L'utente viene reindirizzato alla pagina di visualizzazione del progetto.
    \end{enumerate}
    \item \textbf{Estensioni}: UC12 - Visualizzazione errore generazione progetto
\end{itemize}

\vspace{0.5cm}
\section*{UC12 - Visualizzazione errore generazione progetto}
\begin{itemize}
    \item \textbf{Attori}: Utente autenticato;
    \item \textbf{Descrizione}: L'utente visualizza un messaggio di errore la generazione del progetto fallisce;
    \item \textbf{Precondizioni}: L'utente sta generando un progetto;
    \item \textbf{Postcondizioni}: Un messaggio di errore viene mostrato all'utente;
    \item \textbf{Flusso principale}:
    \begin{enumerate}
        \item L'utente invia la richiesta all'\gls{llm} per la generazione del progetto;
        \item \gls{llm} non riesce a generare il progetto richiesto;
        \item Il sistema visualizza un messaggio di errore indicando che la generazione del progetto è fallita.
    \end{enumerate}
\end{itemize}

\vspace{0.5cm}  
\section*{UC13 - Rigenerazione completa di un progetto}
\begin{figure}[H]
    \centering
    \includegraphics[scale=0.6]{usecase/uc13.png}
    \caption{UC13 - Rigenerazione completa di un progetto}
    \label{fig:uc13}
\end{figure}
\begin{itemize}
    \item \textbf{Attori}: Utente autenticato, \gls{llm};
    \item \textbf{Descrizione}: L'utente rigenera un intero progetto, sostituendo completamente il contenuto esistente;
    \item \textbf{Precondizioni}: 
    \begin{itemize}
        \item L'utente è autenticato;
        \item L'utente si trova nella pagina di visualizzazione del singolo progetto.
    \end{itemize}
    \item \textbf{Postcondizioni}: Il progetto viene completamente rigenerato;
    \item \textbf{Flusso principale}:
    \begin{enumerate}
        \item L'utente seleziona il pulsante di rigenerazione completa del progetto;
        \item Il sistema richiede all'utente l'inserimento di un prompt su cui si baserà la rigenerazione;
        \item L'utente richiede la rigenerazione del progetto;
        \item Il sistema invia la richiesta di rigenerazione al \gls{llm};
        \item Il \gls{llm} rigenera completamente il progetto;
        \item Il progetto rigenerato viene visualizzato all'utente.
    \end{enumerate}
    \item \textbf{Estensioni}: UC12 - Visualizzazione errore generazione progetto
\end{itemize}

\vspace{0.5cm}  
\section*{UC14 - Rigenerazione di un singolo capitolo di un progetto}
\begin{figure}[H]
    \centering
    \includegraphics[scale=0.6]{usecase/uc14.png}
    \caption{UC14 - Rigenerazione di un singolo capitolo di un progetto}
    \label{fig:uc14}
\end{figure}
\begin{itemize}
    \item \textbf{Attori}: Utente autenticato, \gls{llm};
    \item \textbf{Descrizione}: L'utente rigenera un singolo capitolo del progetto, lasciando invariato il resto del progetto;
    \item \textbf{Precondizioni}: L'utente è autenticato e si trova nella pagina di visualizzazione del singolo progetto;
    \item \textbf{Postcondizioni}: Il capitolo selezionato viene rigenerato;
    \item \textbf{Flusso principale}:
    \begin{enumerate}
        \item L'utente seleziona il capitolo da rigenerare;
        \item Il sistema richiede all’utente l’inserimento di un \textit{prompt} su cui si baserà la rigenerazione;
        \item L’utente richiede la rigenerazione del capitolo;
        \item Il sistema invia la richiesta di rigenerazione al \gls{llm} per il capitolo selezionato;
        \item Il \gls{llm} rigenera solo il capitolo selezionato;
        \item Il progetto rigenerato viene visualizzato all'utente.
    \end{enumerate}
    \item \textbf{Estensioni}: UC12 - Visualizzazione errore generazione progetto
\end{itemize}

\vspace{0.5cm}  
\section*{UC15 - Eliminazione di un progetto}
\begin{itemize}
    \item \textbf{Attori}: Utente autenticato;
    \item \textbf{Descrizione}: L'utente elimina un progetto esistente;
    \item \textbf{Precondizioni}: L'utente è autenticato e si trova nella pagina di visualizzazione del singolo progetto;
    \item \textbf{Postcondizioni}: Il progetto viene eliminato dal sistema;
    \item \textbf{Flusso principale}:
    \begin{enumerate}
        \item L'utente seleziona il pulsante per l’eliminazione del progetto;
        \item L'utente conferma l'eliminazione del progetto;
        \item Il sistema elimina il progetto dal \textit{database}.
    \end{enumerate}
\end{itemize}

\vspace{0.5cm}  
\section*{UC16 - Visualizzazione del progetto in formato PDF}
\begin{itemize}
    \item \textbf{Attori}: Utente autenticato;
    \item \textbf{Descrizione}: L'utente visualizza il progetto generato in formato PDF;
    \item \textbf{Precondizioni}: L'utente è autenticato e si trova nella pagina di visualizzazione del singolo progetto;
    \item \textbf{Postcondizioni}: Il progetto viene visualizzato come PDF;
    \item \textbf{Flusso principale}:
    \begin{enumerate}
        \item L'utente seleziona il pulsante di visualizzazione del progetto in formato PDF;
        \item Il sistema mostra il progetto in formato PDF.
    \end{enumerate}
\end{itemize}

\vspace{0.5cm}  
\section*{UC17 - \textit{Download} del progetto in formato PDF}
\begin{itemize}
    \item \textbf{Attori}: Utente autenticato;
    \item \textbf{Descrizione}: L'utente scarica il progetto in formato PDF;
    \item \textbf{Precondizioni}: L’utente si trova nella pagina di visualizzazione del singolo progetto ed ha selezionato il pulsante di visualizzazione del PDF;
    \item \textbf{Postcondizioni}: Il progetto in formato PDF viene scaricato nel dispositivo dell'utente;
    \item \textbf{Flusso principale}:
    \begin{enumerate}
        \item L’utente seleziona il pulsante di \textit{download} del progetto in formato PDF;
        \item L'utente scarica il progetto in formato PDF.
    \end{enumerate}
\end{itemize}

\vspace{0.5cm}  
\section*{UC18 - Visualizzazione delle versioni precedenti di un progetto}
\begin{itemize}
    \item \textbf{Attori}: Utente autenticato;
    \item \textbf{Descrizione}: L'utente visualizza una lista delle versioni precedenti di un progetto selezionato e può accedere ai dettagli di una specifica versione;
    \item \textbf{Precondizioni}: L'utente è autenticato e il progetto selezionato ha almeno due versioni salvate;
    \item \textbf{Postcondizioni}: 
    \begin{itemize}
        \item Il sistema mostra la lista delle versioni precedenti del progetto, inclusi dettagli come numero di versione e data di creazione;
        \item L'utente può selezionare una versione specifica e visualizzarne i dettagli.
    \end{itemize}
    \item \textbf{Flusso principale}:
    \begin{enumerate}
        \item L'utente accede alla pagina dei dettagli di un progetto;
        \item L'utente seleziona la versione del progetto;
        \item Il sistema recupera e mostra una lista delle versioni precedenti del progetto, con i seguenti dettagli:
        \begin{itemize}
            \item Numero di versione;
            \item Data di creazione.
        \end{itemize}
        \item L'utente seleziona una specifica versione dalla lista;
        \item Il sistema mostra i dettagli della versione selezionata, inclusi i contenuti del progetto in quella versione.
    \end{enumerate}
\end{itemize}

\vspace{0.5cm}  
\section*{UC19 - Rimozione di capitoli non necessari da un progetto}
\begin{itemize}
    \item \textbf{Attori}: Utente autenticato;
    \item \textbf{Descrizione}: L'utente rimuove uno o più capitoli non necessari da un progetto per semplificarne la struttura o adattarlo a nuove esigenze;
    \item \textbf{Precondizioni}: 
    \begin{itemize}
        \item L'utente è autenticato;
        \item Il progetto contiene capitoli modificabili o rimovibili.
    \end{itemize}
    \item \textbf{Postcondizioni}: 
    \begin{itemize}
        \item Il sistema aggiorna la struttura del progetto eliminando i capitoli selezionati;
        \item La versione aggiornata del progetto è salvata nel sistema.
    \end{itemize}
    \item \textbf{Flusso principale}:
    \begin{enumerate}
        \item L'utente accede alla pagina di visualizzazione di un progetto;
        \item L'utente seleziona l'opzione "Gestione capitoli";
        \item Il sistema mostra la lista dei capitoli attualmente presenti nel progetto;
        \item L'utente seleziona uno o più capitoli che desidera rimuovere;
        \item L'utente conferma l'operazione di rimozione;
        \item Il sistema elimina i capitoli selezionati e aggiorna il progetto.
    \end{enumerate}
\end{itemize}

\subsection{Tracciamento dei requisiti}
\label{subsec:requisiti}

Il tracciamento dei requisiti è un processo fondamentale per garantire che tutte le funzionalità e le caratteristiche desiderate del sistema siano correttamente identificate e documentate. \\
Per questo progetto, ho derivato i requisiti da due fonti principali: i casi d'uso e le fonti interne al progetto, come documenti tecnici, normative aziendali e linee guida.\\ 

\noindent Ho catalogato e numerato i requisiti funzionali e non funzionali per garantire una chiara tracciabilità durante lo sviluppo del sistema.\\
Ho inoltre associato ad ogni requisito una priorità che indica l'importanza relativa rispetto alle esigenze del progetto.\\

\noindent La nomenclatura dei requisiti segue il formato descritto di seguito:

\begin{center}
\textbf{R[Tipologia][Priorità]-[Numero]}
\end{center}

Dove:
\begin{itemize}
    \item \textbf{R}: Acronimo di \textit{Requisito};
    \item \textbf{Tipologia}: La tipologia del requisito è definita da una delle seguenti lettere:
        \begin{itemize}
            \item \textbf{F} (Funzionale): Requisiti relativi alle funzionalità che il sistema deve fornire;
            \item \textbf{NF} (Non Funzionale): Requisiti che descrivono le qualità del sistema, come le prestazioni, la sicurezza, la scalabilità, ecc.
        \end{itemize}
    \item \textbf{Priorità}: La priorità del requisito è indicata con una delle seguenti lettere:
        \begin{itemize}
            \item \textbf{O} (Obbligatorio): Il requisito è essenziale per il successo del progetto e deve essere soddisfatto senza eccezioni;
            \item \textbf{F} (Facoltativo): Il requisito è desiderato, ma non è essenziale per il funzionamento del sistema;
            \item \textbf{D} (Desiderabile): Il requisito è preferibile ma può essere rinviato o rimosso senza compromettere gravemente il progetto.
        \end{itemize}
    \item \textbf{Numero}: Un numero intero progressivo che identifica in modo univoco ogni requisito.
\end{itemize}

\noindent In tutto ho individuato 19 requisiti funzionali e 3 requisiti non funzionali.\\
Di seguito mostro i più significativi.
\renewcommand{\arraystretch}{1.5} % Increases the row height for better vertical space

\begin{longtable}{|c|>{\centering\arraybackslash}p{0.7\textwidth}|c|} % Adjust the column width with p{0.7\textwidth}
    \hline
    \rowcolor{green!30} % Header color: light green
    \textbf{Codice Requisito} & \textbf{Descrizione} & \textbf{Fonte} \\
    \hline
    \endfirsthead % Start of the table, first header
    
    \hline
    \rowcolor{green!30} % Header color on subsequent pages
    \textbf{Codice Requisito} & \textbf{Descrizione} & \textbf{Fonte} \\
    \hline
    \endhead % Continuation of the table (repeats on every page)
    
    \hline
    \multicolumn{3}{|c|}{\rowcolor{green!30} \textbf{Requisiti Funzionali}} \\
    \hline % Adds a horizontal line after the header row
    \textbf{RFO-1} & Il sistema deve consentire all'utente non autenticato di effettuare il \textit{login} inserendo \textit{email} e \textit{password} & UC1 \\
    \hline
    \textbf{RFO-2} & Il sistema deve autenticare l'utente con credenziali valide e reindirizzarlo alla \textit{dashboard} & UC1 \\
    \hline
    \textbf{RFO-3} & Il sistema deve mostrare un messaggio di errore in caso di \textit{login} fallisca & UC3 \\
    \hline
    \textbf{RFO-4} & Il sistema deve consentire all'utente autenticato di effettuare il logout & UC2 \\
    \hline
    \textbf{RFO-5} &  Il sistema deve mostrare la lista di tutti i progetti disponibili all'utente autenticato nella \textit{dashboard} & UC4 \\
    \hline
    \textbf{RFD-6} & Il sistema deve permettere di cercare e filtrare i progetti in base a nome o data specificati dall’utente autenticato & UC5 \\
    \hline
    \textbf{RFO-7} & Il sistema deve consentire la visualizzazione dettagliata di un singolo progetto & UC6 \\
    \hline
    \textbf{RFO-8} & Il sistema deve consentire la visualizzazione della descrizione dettagliata di un \textit{preset} & UC7 \\
    \hline
    \textbf{RFD-9} & Il sistema deve permettere il salvataggio di un \textit{preset} parzialmente compilato come bozza & UC8 \\
    \hline
    \textbf{RFD-10} & Il sistema deve mostrare la lista di tutte le bozze disponibili all'utente autenticato & UC9 \\
    \hline
    \textbf{RFD-11} & Il sistema deve permettere di cercare e filtrare le bozze in base a nome o data specificati dall’utente autenticato & UC10 \\
    \hline
    \textbf{RFO-12} & Il sistema deve consentire all'utente di generare un progetto utilizzando un \textit{preset} compilato nella sua interezza & UC11 \\
    \hline
    \textbf{RFO-13} & Il sistema deve consentire all'utente di rigenerare un intero progetto, sostituendo completamente il contenuto esistente & UC12 \\
    \hline
    \textbf{RFD-14} & Il sistema deve consentire all'utente di rigenerare un singolo capitolo di un progetto lasciando invariati i restanti capitoli & UC13 \\
    \hline
    \textbf{RFO-15} & Il sistema deve consentire all'utente di eliminare un progetto esistente & UC14 \\
    \hline
    \textbf{RFO-16} & Il sistema deve consentire la visualizzazione di un progetto in formato PDF & UC15 \\
    \hline
    \textbf{RFO-17} & Il sistema deve consentire il download di un progetto generato in formato PDF & UC16 \\
    \hline
    \textbf{RFF-18} & Il sistema deve consentire la visualizzazione delle versioni precedenti di un progetto & UC17 \\
    \hline
    \textbf{RFF-19} & Il sistema deve consentire la rimozione di capitoli non necessari dal progetto generato & UC18 \\
    \hline
    \pagebreak
    \hline
    \multicolumn{3}{|c|}{\rowcolor{green!30} \textbf{Requisiti Non Funzionali}} \\
    \hline % Adds a horizontal line after the subtitle
    \textbf{RNFO-1} & Il sistema deve garantire che i tempi di risposta per la generazione o la rigenerazione di un progetto non superino i 30 secondi in condizioni di carico normale & Interna \\
    \hline
    \textbf{RNFO-2} & Il sistema deve fornire un’interfaccia utente \textit{responsive}, accessibile da dispositivi \textit{desktop} e \textit{mobile} & Interna \\
    \hline
    \textbf{RNFO-3} & I messaggi di errore devono essere chiari, concisi e facilmente comprensibili dall'utente & Interna \\
    \hline
    \multicolumn{3}{|c|}{\rowcolor{green!30} \textbf{Requisiti Tecnologici}} \\
    \hline
    \textbf{RT-1} & Il sistema deve utilizzare \textit{AWS Bedrock} per la generazione e rigenerazione di contenuti dei progetti & Interna \\
    \hline
    \textbf{RT-2} & Il sistema deve supportare la generazione di \textit{file} PDF utilizzando \textit{Puppeteer} & Interna \\
    \hline
    \textbf{RT-3} & Il frontend dell’applicazione deve utilizzare la libreria \textit{ReactJS} & Interna \\
    \hline
    \textbf{RT-4} & Il backend dell’applicazione deve utilizzare \textit{NestJS} & Interna \\
    \hline
    \textbf{RT-5} & Il sistema deve essere autenticato tramite \textit{AWS Cognito} & Interna \\
    \hline
    \textbf{RT-6} & Il sistema deve utilizzare \textit{AWS S3} per l’archiviazione dei documenti PDF. & Interna \\
    \hline
    \textbf{RT-7} & Le \gls{api} devono essere autenticate tramite \gls{jwt}. & Interna \\
    \hline
    \textbf{RT-8} & Il \textit{database} utilizzato deve essere \textit{MongoDB}. & Interna \\
    
    \hline
    \caption{Tracciamento dei requisiti} % Table caption
    \label{tab:requisiti-stage} % Label for referencing the table
\end{longtable}