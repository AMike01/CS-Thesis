\subsection{Scelte tecnologiche}
\label{subsec:scelte-tecnologiche}

La scelta delle tecnologie utilizzate nel progetto è stata in gran parte imposta dall'azienda.\\
In particolare, l'azienda ha imposto l'adozione di \textit{ReactJS} come \textit{framework} per il \gls{frontend}, di \textit{NestJS} per il \gls{backend} e di \textit{MongoDB} come \textit{database}.\\
Queste scelte sono state motivate da una serie di considerazioni interne, tra cui la familiarità aziendale con le tecnologie e la loro capacità di soddisfare le esigenze di scalabilità e manutenibilità del progetto.\\

\noindent Inoltre, l'azienda ha imposto l'utilizzo di una serie di servizi \gls{aws} per la gestione dell'autenticazione, la generazione dei progetti e il salvataggio dei documenti.\\
Questi sono stati ampiamente discussi nella {\hyperref[sez:tecnologie-sviluppo]{Sezione 1.4}} relativa alle tecnologie di sviluppo.

\subsubsection{Utilizzo di \textit{LangChain} per la generazione dei progetti}

Inizialmente, avevo deciso di utilizzare le \gls{api} standard di \textit{AWS Bedrock} per la generazione dei progetti, ma in una prima iterazione di sviluppo non è stato possibile creare i \textit{mockup} di \textit{Bedrock} necessari per il \textit{testing}.\\

\noindent Avendo sviluppato solamente una versione iniziale della funzione di generazione dei progetti, per superare questa difficoltà, ho deciso di impiegare \textit{LangChain}, una libreria che fornisce un'interfaccia più flessibile per interagire con modelli linguistici come quelli di \textit{AWS Bedrock}.\\

\noindent La scelta di \textit{LangChain} è stata motivata anche dal fatto che essa consente di imporre un \textit{output strutturato} all'\gls{llm}, che permette quindi di garantire risultati più costanti e prevedibili rispetto alle risposte meno formattate ottenute tramite le \gls{api} di \textit{Bedrock}.\\

\noindent Questa scelta ha migliorato la gestione dei dati e la qualità dei progetti generati, facilitando la creazione di un flusso di lavoro più stabile.

\subsubsection{Generazione dei PDF nel \gls{backend}}

Un'altra importante scelta tecnologica riguarda la generazione dei \textit{file} PDF.\\
Dopo varie ricerche \textit{online} ed una discussione con il \textit{tutor} aziendale, ho deciso di gestire la generazione dei PDF nel \gls{backend}, piuttosto che nel \gls{frontend}.\\

\noindent Ho preso questa decisione per diverse ragioni: in primo luogo, la generazione del PDF può richiedere un'elaborazione significativa e, generandolo nel \gls{backend}, è possibile ridurre il carico sul \gls{frontend}, migliorando così le performance e la reattività dell'applicazione.\\
Inoltre, il \gls{backend} è in grado di gestire più facilmente il salvataggio dei PDF su \textit{AWS S3}, centralizzando la gestione dei \textit{file} e semplificando l'archiviazione ed il recupero dei documenti.\\

\noindent Questo approccio ha anche semplificato la sicurezza ed il controllo sui documenti generati, poiché il processo di generazione e salvataggio avviene su un \textit{server} dedicato, riducendo il rischio di problemi legati alla gestione dei \textit{file} a livello \textit{client}.
