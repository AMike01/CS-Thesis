\pagebreak
\section{Validazione}
\label{sez:validazione}

La validazione è il processo che verifica se il prodotto finale soddisfa i requisiti e le aspettative degli utenti o delle parti interessate. \\
A differenza della verifica, che si concentra sull’accuratezza e completezza tecnica delle singole componenti (ad esempio, controllando che il codice sia conforme alle specifiche), la validazione si focalizza sull’effettivo valore e utilità del \textit{software} nel contesto operativo previsto.  \\

\subsection{Test di accettazione}
\label{subsec:test-accettazione}

I \textit{test} di accettazione sono stati condotti in collaborazione con il \textit{tutor} aziendale, con l’obiettivo di verificare che il progetto soddisfacesse i requisiti funzionali e non funzionali definiti in fase di analisi.\\

\noindent Durante la sessione di \textit{test}, il tutor aziendale ha valutato il comportamento del sistema rispetto alle aspettative aziendali e alle specifiche iniziali. La valutazione finale è stata positiva, confermando che il \textit{software} è pronto per essere utilizzato nell’ambiente di produzione e soddisfa pienamente le esigenze del progetto.
\subsection{Presentazione del progetto}
\label{subsec:presentazione-progetto}

L'ultima settimana di \textit{stage} mi è stata richiesta dall'azienda una presentazione finale del progetto ai colleghi aziendali, con l'obiettivo di illustrare il lavoro svolto, i problemi riscontrati durante lo sviluppo e le possibili migliorie da apportare in futuro.\\

\noindent Durante la presentazione ho spiegato lo scopo del progetto e le tecnologie utilizzate, evidenziando il valore aggiunto che potrebbe apportare all'azienda, ho inoltre mostrato una \textit{demo} del prodotto per dimostrare le sue funzionalità in tempo reale.\\  

\noindent La presentazione è stata accolta in modo molto positivo dai partecipanti. \\Al termine, sono state poste alcune domande per chiarire alcuni dettagli implementativi del progetto. In particolare, un collega ha manifestato interesse riguardo alla creazione dei \gls{prompt} utilizzati per la generazione dei progetti ed il loro processo di salvataggio, avviando un confronto costruttivo sull’argomento.\\
Questo ha confermato l’interesse e la rilevanza del lavoro svolto all'interno del contesto aziendale.

