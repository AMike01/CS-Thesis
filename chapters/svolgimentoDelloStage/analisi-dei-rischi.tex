\pagebreak
\section{Analisi dei rischi}
\label{sez:analisi-dei-rischi}

L'analisi dei rischi è un'attività essenziale per identificare e gestire eventi incerti che potrebbero compromettere il successo del progetto. \\
Consiste nell’individuare i rischi, valutarne la probabilità e l’impatto, e pianificare strategie di mitigazione per ridurne gli effetti.\\

\noindent Per ogni rischio stilato nella {\hyperref[tab:analisi-rischi]{Tabella 3.1}} vengono definiti:

\begin{itemize}
    \item \textbf{Probabilità di occorrenza:} quanto è probabile che il rischio si verifichi (\textit{alta, media o bassa});
    \item \textbf{Descrizione:} una spiegazione del rischio e delle sue conseguenze;
    \item \textbf{Mitigazione:} azioni preventive o correttive per ridurre l’impatto.
\end{itemize}

\renewcommand{\arraystretch}{1.5} % Increases the row height for better vertical space

\begin{longtable}{|c|>{\centering\arraybackslash}p{0.7\textwidth}|} % Adjust the column width with p{0.7\textwidth}
    \hline
    \rowcolor{green!30} 
    \textbf{\#1} & \textbf{Inesperienza tecnologica} \\
    \hline
    \textbf{Occorrenza} & Alta \\
    \hline
    \textbf{Descrizione} & Lo stagista possiede poca esperienza con le tecnologie e gli strumenti principali necessari per il progetto, il che potrebbe rallentare lo sviluppo e compromettere la qualità delle soluzioni realizzate\\
    \hline
    \textbf{Mitigazione} & Prevedere una settimana iniziale dedicata all’autoformazione e all’approfondimento guidato delle tecnologie chiave. Durante questa fase, verranno forniti materiali di studio e accesso a risorse dedicate, con il supporto del \textit{tutor} aziendale per rispondere a eventuali dubbi e garantire un apprendimento efficace\\
    \hline

    \rowcolor{green!30} % Header color: light green
    \textbf{\#2} & \textbf{Inesperienza organizzativa} \\
    \hline
    \textbf{Occorrenza} & Media \\
    \hline
    \textbf{Descrizione} & Ridotta esperienza nella realizzazione di progetti utilizzando metodologie agili e nella gestione del lavoro per obiettivi. Questo potrebbe portare a inefficienze, difficoltà nel rispettare le scadenze e mancato raggiungimento degli obiettivi prefissati\\
    \hline
    \textbf{Mitigazione} & Fornire formazione iniziale sulle metodologie agili e affiancare una figura esperta per guidare lo stagista. Utilizzare strumenti di \textit{project management} per facilitare pianificazione e monitoraggio, organizzando retrospettive periodiche per migliorare il processo\\
    \hline
    \pagebreak

    \hline
    \rowcolor{green!30} % Header color: light green
    \textbf{\#3} & \textbf{Complessità del contesto di uso di servizi di \gls{generative-ai}} \\
    \hline
    \textbf{Occorrenza} & Alta \\
    \hline
    \textbf{Descrizione} & L’esperienza con i servizi di \gls{generative-ai} è limitata ad un contesto accademico e non è mai stata applicata in un progetto reale, aumentando il rischio di errori nella configurazione e integrazione\\
    \hline
    \textbf{Mitigazione} & Prevedere una fase iniziale dedicata allo studio della documentazione ufficiale dei servizi di \gls{generative-ai} utilizzati nel progetto, con approfondimenti su configurazione e integrazione. Integrare il percorso con il supporto di risorse esterne o \textit{community} online per risolvere eventuali problematiche tecniche\\
    \hline

    \rowcolor{green!30} % Header color: light green
    \textbf{\#4} & \textbf{Ridotta conoscenza dei servizi \textit{cloud} \gls{aws}} \\
    \hline
    \textbf{Occorrenza} & Media \\
    \hline
    \textbf{Descrizione} & Lo stagista ha una conoscenza limitata dei servizi \textit{cloud} \gls{aws}, il che potrebbe causare difficoltà nell’implementazione e gestione delle infrastrutture richieste\\
    \hline
    \textbf{Mitigazione} & Prevedere lo studio della documentazione ufficiale di \gls{aws} con particolare attenzione ai servizi rilevanti per il progetto. Affiancare questa attività con il supporto e l’aiuto di colleghi più esperti per chiarire dubbi e fornire indicazioni pratiche sull’utilizzo dei servizi\\
    \hline
    
    \rowcolor{green!30} % Header color: light green
    \textbf{\#5} & \textbf{Assenze per motivi di salute e/o personali} \\
    \hline
    \textbf{Occorrenza} & Bassa \\
    \hline
    \textbf{Descrizione} &  Lo stagista potrebbe dover affrontare assenze per motivi di salute o personali, con il rischio di rallentare l'avanzamento del progetto\\
    \hline
    \textbf{Mitigazione} &In caso di assenza programmata o imprevista, lo stagista dovrà avvisare tempestivamente il tutor aziendale. Questo permetterà di pianificare adeguatamente la gestione delle attività e delle scadenze, riducendo al minimo l'impatto sul progresso del progetto\\
    \hline

    \caption{Analisi dei rischi del progetto} % Table caption
    \label{tab:analisi-rischi} % Label for referencing the table
\end{longtable}