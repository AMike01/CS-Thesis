\subsection{\textit{Best practices}}
\label{sez:best-practices}

Durante la fase di sviluppo del progetto ho adottato diverse \textit{best practices} di codifica, tra cui:

\begin{itemize}
    \item \textbf{\textit{Arrow Functions}}: Utilizzo consistente delle \textit{arrow functions} per una sintassi più concisa e un \textit{binding} più chiaro del \texttt{this}:
    \begin{verbatim}
    const add = (a, b) => a + b;
     \end{verbatim}
    \item \textbf{Convenzioni di \textit{naming}}: 
    \begin{itemize}
        \item \textit{PascalCase} per la definizione dei \textit{file ReactJs} e \textit{NestJs};
        \item \textit{camelCase} per variabili e funzioni;
        \item \textit{UPPER\_CASE} per le costanti.
    \end{itemize}

    \item \textbf{Creazione di un \textit{file .tsx} per ogni componente \textit{ReactJs}}:\\
    Questo approccio facilita i cambiamenti senza dover modificare il codice in più file. 
    Ogni componente \textit{ReactJs} ha il proprio \textit{file .tsx}, migliorando la modularità e la manutenibilità del codice.\\
    Ad esempio, un componente ‘\textit{SnackBar}‘ sarà definito in \textit{SnackBar.tsx}‘;
    
    \item \textbf{Utilizzo di \gls{git-flow}}: Struttura di \textit{branching} per una gestione più efficiente del ciclo di vita del software.\\
    Ampiamente discussa nella {\hyperref[subsec:vincoli-organizzativi]{Sezione 2.5.2}} relativa ai vincoli organizzativi;

    \item \textbf{\textit{Code Review}}: Ogni modifica al codice deve essere sottoposta a \textit{code review} da parte di un altro membro del \textit{team}.\\
    Questo processo aiuta a individuare errori, migliorare la qualità del codice e condividere conoscenze tra i membri del \textit{team};    
\end{itemize}

