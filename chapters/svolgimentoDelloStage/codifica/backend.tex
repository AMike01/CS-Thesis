\subsection{\gls{backend}}
\label{sez:backend}

La struttura della \textit{repository} \gls{backend} è la seguente:

\begin{itemize}
    \item \textbf{\textit{src}}: Contiene varie cartelle organizzate per funzionalità, tra cui:
    \begin{itemize}
        \item \textbf{\textit{auth}}: Contiene i \textit{middleware} per l'autenticazione delle \gls{api};
        \item \textbf{\textit{aws-bedrock}}: Contiene la logica della generazione dei progetti;
        \item \textbf{\textit{pdf}}: Contiene la logica per la generazione e salvataggio su \textit{AWS S3} dei documenti PDF;
        \item \textbf{\textit{utils}}: Contiene tutti i \textit{mock} utili per il \textit{testing} o funzioni di supporto globale.
    \end{itemize}
    \item \textbf{\textit{package.json}}: Contiene le dipendenze del progetto;
    \item \textbf{\textit{DockerFile}}: Contiene il file \textit{Docker} per la creazione del \gls{container} per \textit{MongoDB};
\end{itemize}

\noindent Per ogni cartella all'interno di \textit{src} (ad esclusione delle cartelle \textbf{\textit{auth}} e \textbf{\textit{utils}}), si ha una struttura comune che include:
\begin{itemize} 
    \item una cartella \textit{dto} contenente i \gls{dtog} relativi alla funzionalità;
    \item una cartella \textit{schemas} contenente lo schema \gls{odmg} relativo all'entità gestita;
    \item un \textit{controller} per la gestione delle richieste \gls{http};
    \item un \textit{module} per l'iniezione delle dipendenze;
    \item un \textit{service} per la logica di \textit{business};
    \item i \textit{test} dei \textit{controller} e dei \textit{service}.
\end{itemize}

\subsubsection{\textit{Endpoints} delle \gls{api}}

\noindent Nel \gls{backend} dell'applicazione, ho progettato e sviluppato un totale di 5 \textit{endpoints}, che complessivamente implementano 11 \gls{api}, ognuna delle quali svolge un ruolo fondamentale nella gestione dei dati e nell'esecuzione delle operazioni richieste dall'applicazione. \\
Questi \textit{endpoints} sono stati creati per garantire una comunicazione fluida e sicura tra il \gls{frontend}, il \gls{backend} e i servizi esterni, assicurando che ogni componente del sistema possa interagire in modo efficiente.\\
Inoltre, ho definito 8 \gls{dto}, che sono essenziali per la corretta gestione, validazione e trasmissione dei dati tra le diverse componenti del sistema, permettendo di mantenere l'integrità e la coerenza delle informazioni.\\

\noindent Di seguito, approfondisco gli \textit{endpoints} più rilevanti, focalizzandomi sulle funzionalità avanzate che implementano e analizzando le modalità con cui questi \textit{endpoints} interagiscono con il \gls{frontend} ed i servizi esterni, garantendo una gestione ottimale delle risorse e una comunicazione efficiente tra le varie parti del sistema. \\

\subsection*{\texttt{/aws-bedrock}}
Ho implementato l’\textit{endpoint} \texttt{/aws-bedrock} per gestire la generazione automatica dei progetti sfruttando le capacità di \textit{AWS Bedrock}.\\
La sua funzione principale è quella di orchestrare la comunicazione tra il sistema e un modello di linguaggio (\gls{llm}) avanzato, consentendo la creazione di progetti dettagliati.\\

\noindent Per garantire consistenza e affidabilità nel risultato, ho utilizzato la libreria \textit{LangChain}, un \textit{framework} che supporta la definizione di output strutturati direttamente dall’\gls{llm},
permettendo di definire uno schema di \textit{output} prevedibile.\\

\pagebreak
\noindent La {\hyperref[lst:funzione-generazione-progetto]{\textit{Listing} 3.2}} mostra la dichiarazione della \gls{llm} e la funzione di generazione dei progetti implementata nel \textit{service}, che si occupa di invocare il modello di linguaggio e di restituire il risultato al \gls{frontend}.

\begin{lstlisting}[caption={Dichiarazione \gls{llm} e sua invocazione}, label={lst:funzione-generazione-progetto}]
private readonly llm: ChatBedrockConverse;
  constructor() {  
    // declare the LLM
    this.llm = new ChatBedrockConverse({
      model: "anthropic.claude-3-5-sonnet-20241022-v1:5",
      temperature: 0.3,
      topP: 0.9,
      maxTokens: 4096,
      region: "eu-central-1",
      // AWS credentials
      credentials: fromIni({ profile: "dev" }), 
    });
  }
  async generateProject(
    prompt: string,
    promptId: string,
    projectId: string
  ): Promise<CreateProjectDTO> {

    try {
      // Define the output schema
      const structuredLlm = this.llm.withStructuredOutput(ProjectSchema); 
      // Invoke the LLM
      const response = await structuredLlm.invoke(prompt); 
      const result : CreateProjectDTO = response;

      // Assign the promptId
      result.promptId = promptId; 
      // Assign the projectId
      result.projectId = projectId; 

      return result;
    } catch (error) {
      throw new Error(
        `Failed to generate project with AWS Bedrock: ${error.message}`
      );
    }
  }
\end{lstlisting}
\pagebreak
\subsection*{\texttt{/pdf}}
Ho creato l'\textit{endpoint} \texttt{/pdf} per gestire la generazione dei documenti PDF ed il loro salvataggio su \textit{AWS S3} a partire dai progetti creati.\\
Questo \textit{endpoint} svolge un ruolo fondamentale nel flusso di lavoro complessivo, poiché consente agli utenti di ottenere una rappresentazione visiva e condivisibile dei progetti generati.\\
Una volta creato, il PDF viene caricato su un \textit{bucket} \textit{AWS S3}, assicurandone la memorizzazione a lungo termine e l’accessibilità tramite un \textit{URL} univoco.\\ 

\noindent La {\hyperref[lst:funzione-generazione-pdf]{\textit{Listing} 3.3}} riporta la funzione di generazione del PDF, si noti la popolazione del template \gls{html} con i dati del progetto e il salvataggio del PDF generato su \textit{AWS S3}.
\begin{lstlisting}[caption={Funzione di generazione del PDF, con salvataggio su \textit{AWS S3}}, label={lst:funzione-generazione-pdf}]
async generatePdf(user: User, project: Project): Promise<string> {
    // Populate the HTML template    
    const content = await this.populateTemplate(project);  
    // Get the browser instance
    const browser: Browser = await this.getBrowser(); 

    const page: Page = await browser.newPage();
    // Set the content of the page
    await page.setContent(content); 

    try {
      // Generate the PDF
      const pdfBuffer: Uint8Array = await page.pdf({
        format: "A4",
        printBackground: true,
      }); 
      // Save the PDF on AWS S3
      this.savePdf(user, pdfBuffer, project.projectId, project.projectName, project.version);  
      // Return the signed URL to the frontend
      return this.getSignedUrl(user, project.projectId, project.projectName, project.version);
    } catch (error) {
      throw new Error("Failed to generate PDF");
    } finally {
      await page.close();
      await this.closeBrowser();
    }
  }
\end{lstlisting}