\subsection{\textit{Frontend}}
\label{sez:frontend}

La struttura della \textit{repository} \gls{frontend} è la seguente:

\begin{itemize}
    \item \textbf{\textit{public}}: Contiene le immagini o loghi utilizzati nell'applicazione;
    \item \textbf{\textit{src}}: Contiene il codice sorgente dell'applicazione, suddiviso in diverse cartelle:
    \begin{itemize}
        \item \textbf{\textit{tests}}: Contiene tutti i \textit{test} unitari creati per verificare il corretto funzionamento delle varie parti del codice;
        \item \textbf{\textit{components}}: Contiene tutti i componenti creati, che sono le unità riutilizzabili dell'interfaccia utente;
        \item \textbf{\textit{hooks}}: Contiene i \textit{custom hooks} creati, che sono funzioni speciali di \textit{React} per gestire lo stato e altri effetti collaterali;
        \item \textbf{\textit{interfaces}}: Contiene le interfacce create, che definiscono i tipi di dati utilizzati nel progetto;
        \item \textbf{\textit{pages}}: Contiene le pagine vere e proprie, che rappresentano le diverse viste dell'applicazione, come la pagina di \textit{login} e la \textit{dashboard};
        \item \textbf{\textit{routes}}: Contiene le \textit{routes} del progetto, che definiscono la navigazione tra le diverse pagine e gestiscono l'autorizzazione;
        \item \textbf{\textit{services}}: Contiene i servizi, in particolare quelli di autenticazione, che gestiscono la logica di \textit{business} e le chiamate \gls{api};
        \item \textbf{\textit{app.tsx}}: Il \textit{file} principale dell'applicazione che definisce la struttura generale e include i componenti principali;
        \item \textbf{\textit{main.tsx}}: Il \textit{file} di ingresso dell'applicazione che avvia \textit{React} e renderizza l'applicazione nel \textit{DOM};
    \end{itemize}
    \item \textbf{\textit{package.json}}: Contiene le dipendenze del progetto;
    \item \textbf{\textit{tsconfig.json}}: Contiene le configurazioni del compilatore \textit{TypeScript}.
\end{itemize}

\noindent Vado ora a descrivere le pagine che ho implementato nel \gls{frontend} dell'applicazione, con una panoramica delle loro funzionalità e delle loro interazioni con il \gls{backend}.

\subsubsection{Pagina di \textit{Login}}
La pagina di \textit{login} rappresenta la prima pagina con cui gli utenti interagiscono.\\
È stata progettata per garantire un'esperienza utente semplice e intuitiva, rispettando al contempo elevati standard di sicurezza.
\begin{itemize}
    \item \textbf{Funzionalità principali}: La pagina presenta un \textit{form} composto da due campi: \textit{email} e \textit{password}.\\
    I dati inseriti sono sottoposti a validazione lato \textit{client} utilizzando la libreria \textit{Zod} per assicurare che rispettino il formato previsto (ad esempio, che l'\textit{email} sia valida e che la \textit{password} soddisfi determinati requisiti di complessità). \\
    Sono inoltre implementati messaggi di errore per informare l'utente in caso di credenziali errate o di campi mancanti;
    \item \textbf{Interazione con il \gls{backend}}: Una volta completata la validazione lato \textit{client}, i dati vengono inviati al servizio di autenticazione \textit{AWS Cognito} tramite una chiamata \gls{api}, se le credenziali risultatno corrette viene restituito \gls{jwtg} che viene memorizzato nel \textit{local storage} del \textit{browser} per gestire la sessione utente. \\
    In caso di errore, la pagina notifica l'utente mostrando messaggi specifici.
\end{itemize}

\noindent All'interno della {\hyperref[lst:login-frontend]{\textit{Listing} 3.1}} è possibile osservare la funzione di \textit{login} implementata nel \gls{frontend}, che andrà poi a chiamare il servizio di autenticazione \textit{AWS Cognito}.

\begin{lstlisting}[caption={Funzione di \textit{login} nel \gls{frontend}}, label={lst:login-frontend}]
    const login = async (email: string, password: string): Promise<void> => {
        try {
          const cognitoUser: SignInOutput | undefined = await auth.login(
            email,
            password,
          );
          if (cognitoUser !== undefined) {
            setUser(cognitoUser);
            if (
              cognitoUser.nextStep.signInStep ===
              'CONFIRM_SIGN_IN_WITH_NEW_PASSWORD_REQUIRED'
            ) {
              setNewPasswordChallenge(true);
            } else {
              showSnackbar({message: "Login successful!", severity: "success"});
              navigate("/dashboard");
            }
          }
        } catch (err) {
          showSnackbar({message: "The email and/or password are incorrect", severity: "error"});
        }
      };
    \end{lstlisting}

\subsubsection{\textit{Dashboard}: Lista progetti e bozze}
La \textit{dashboard} è il fulcro dell'applicazione, fornendo una panoramica completa dei progetti creati e delle bozze salvate dall'utente.
\begin{itemize}
    \item \textbf{Funzionalità principali}: La \textit{dashboard} è divisa in due sezioni principali.\\
    La prima mostra i progetti generati, con dettagli come il nome del progetto e la data di creazione. La seconda sezione è dedicata alle bozze, consentendo di riprendere un \textit{preset} incompleto in qualsiasi momento.\\
    Sono implementate funzionalità di ricerca e filtri per facilitare la gestione di un gran numero di progetti e/o bozze;
    \item \textbf{Interazione con il \gls{backend}}: Al caricamento della pagina, una chiamata \gls{api} \textit{GET} recupera i progetti  e le bozze dal \gls{backend}, utilizzando l'ID dell'utente per filtrare i risultati.
\end{itemize}

\subsubsection{Pagina di creazione di un progetto}
Questa pagina consente agli utenti di creare nuovi progetti fornendo tutte le informazioni necessarie o, in alternativa, di salvare una bozza per completare il lavoro successivamente. 
\begin{itemize}
    \item \textbf{Funzionalità principali}: La pagina include un \textit{form} dinamico che richiede all'utente di selezionare un \textit{preset} e compilarlo per poter andare a generare un progetto.\\ 
    Durante l'inserimento, viene effettuata una validazione lato \textit{client} per assicurarsi che tutti i campi obbligatori siano compilati e che i dati forniti rispettino i formati richiesti. \\
    L'utente ha la possibilità di salvare il \textit{preset} come bozza o di richiedere la generazione del progetto;
    \item \textbf{Interazione con il \gls{backend}}: Una volta completata la compilazione del \textit{form}, i dati vengono inviati al \gls{backend} tramite una chiamata \gls{api} \textit{POST} che si occupa della generazione del progetto.\\
\end{itemize}

\subsubsection{Pagina di Dettaglio Progetto}
Questa pagina fornisce una vista approfondita di un singolo progetto, mostrando tutte le informazioni associate e consentendo di accedere al documento PDF generato.
\begin{itemize}
    \item \textbf{Funzionalità principali}: La pagina presenta dettagli completi del progetto, inclusi nome, data di creazione e tutti i capitoli che lo compongono.\\
    È possibile visualizzare il PDF generato direttamente all'interno di un visualizzatore integrato oppure di scaricarlo per un utilizzo futuro.\\
    Inoltre, la pagina permette di andare a rigenerare il progetto nella sua totalità o in parte;
    \item \textbf{Interazione con il \gls{backend}}: All'apertura della pagina, una chiamata \gls{api} \textit{GET} recupera tutte le versioni del progetto.\\
    Il PDF, generato precedentemente nel \gls{backend}, è reso disponibile tramite un \textit{URL} fornito da \textit{AWS S3}, garantendo sicurezza e velocità di accesso. 
\end{itemize}

\noindent All'interno della {\hyperref[lst:single-project-page]{\textit{Listing} 3.2}} è possibile osservare la funzione di \textit{retrieve} di tutte le versioni del progetto, si noti la riga \texttt{projectsData[projectsData.length - 1]}, questa va ad impostare come progetto selezionato l'ultima versione disponibile.
\pagebreak
\begin{lstlisting}[caption={Funzione di \textit{retrieve} di tutte le versioni dei progetti}, label={lst:single-project-page}]
  const fetchProject = async () => {
    try {
      const response = await get(`/projects/${id}`);
      if (response && response.data) {
        const projectsData = response.data as Project[];
        setProjects(projectsData);
        setSelectedProject(
          projectsData.length > 0
            ? projectsData[projectsData.length - 1]
            : null
        );
      } else {
        throw new Error("Invalid response format");
      }
      setLoading(false);
    } catch (err) {
      showSnackbar({
        message: "Failed to fetch project details.",
        severity: "error",
      });
      setLoading(false);
    }
  };
  \end{lstlisting}
